\documentclass[a4paper,11pt]{article}

\usepackage[utf8]{inputenc}
\usepackage[english]{babel}
\usepackage{amssymb, amsmath, amsthm, mathrsfs}
\usepackage[left=1.0in,right=1.0in,top=1.0in,bottom=1.0in]{geometry}
\usepackage[T1]{fontenc}
\usepackage{array}
\usepackage{longtable}
\usepackage{multirow}
\usepackage{calc}
\usepackage[inline,shortlabels]{enumitem}
\usepackage{changepage}
\usepackage{booktabs}
\usepackage{capt-of}
\usepackage{subcaption}
\usepackage[leftcaption]{sidecap}
\usepackage[numbers]{natbib}
\usepackage{times}
\usepackage{titlesec}
\usepackage{xcolor}
\usepackage{lineno}
\usepackage{xpatch}
\xpatchcmd\swappedhead{~}{.~}{}{}
\allowdisplaybreaks

\newtheoremstyle{mythm}
{}                % Space above
{}                % Space below
{\itshape}        % Theorem body font % (default is "\upshape")
{1.5em}                % Indent amount
{\scshape}       % Theorem head font % (default is \mdseries)
{.}               % Punctuation after theorem head % default: no punctuation
{0.5em}               % Space after theorem head
{}                % Theorem head spec
\theoremstyle{mythm}


\newtheorem*{theorem*}{Theorem}
\newtheorem{theorem}{Theorem}
\newtheorem{fact}[theorem]{Fact}
\newtheorem{proposition}[theorem]{Proposition}
\newtheorem{lemma}[theorem]{Lemma}
\newtheorem{corollary}[theorem]{Corollary}
\newtheorem{question}[theorem]{Question}
\newtheorem{result}[theorem]{Result}
\newtheorem{observation}[theorem]{Observation}
\newtheorem{conjecture}[theorem]{Conjecture}

\newtheoremstyle{mydef}
{}                % Space above
{}                % Space below
{}        % Theorem body font % (default is "\upshape")
{1.5em}                % Indent amount
{\scshape}       % Theorem head font % (default is \mdseries)
{.}               % Punctuation after theorem head % default: no punctuation
{0.5em}               % Space after theorem head
{}                % Theorem head spec
\theoremstyle{mydef}

\newtheorem{example}[theorem]{Example}
\newtheorem{definition}[theorem]{Definition}
\newtheorem{remark}[theorem]{Remark}
\newtheorem*{remark*}{Remark}

\makeatletter
\renewenvironment{proof}[1][\proofname]{\par
  \pushQED{\qed}%
  \normalfont \topsep6\p@\@plus6\p@\relax
  \trivlist
\item\relax
  {\hspace{1.5em}\itshape
    #1\@addpunct{.}}\hspace\labelsep\ignorespaces
}{%
  \popQED\endtrivlist\@endpefalse
}
\makeatother

\def\Box{\hskip1ex\vbox{\hrule height0.6pt\hbox{%
      \vrule height1.3ex width0.6pt\hskip0.8ex
      \vrule width0.6pt}\hrule height0.6pt
  }}
\renewcommand{\qed}{\Box}

\newcommand{\red}[1]{\textcolor{red}{#1}}
\newcommand{\blue}[1]{\textcolor{blue}{#1}}
\newcommand{\purple}[1]{\textcolor{magenta}{#1}}
\newcommand{\ddet}{\text{det}}
\renewcommand{\pmod}[1]{\text{ (mod $#1$)}}
\newcommand{\mmod}[2]{#1\text{ mod }#2}
\newcommand{\abs}[1]{\left\vert #1 \right\vert}
\newcommand{\C}{\mathbf{C}}
\newcommand{\Z}{\mathbf{Z}}
\newcommand{\N}{\mathbf{N}}
\newcommand{\LL}{\mathscr{G}}
\newcommand{\z}{\mathbin{\ooalign{$\hidewidth i \hidewidth$\cr$\phantom{+}$}}}
\newcommand{\y}{\mathbin{\ooalign{$\hidewidth j \hidewidth$\cr$\phantom{+}$}}}
\newcommand{\gf}{\text{GF}}
\newcommand{\ord}{\text{ord}}

\newcolumntype{R}{>{\scriptsize}r}
\newcolumntype{L}{>{\scriptsize}l}
\newcolumntype{C}{>{\scriptsize}c}

\renewcommand{\citenumfont}[1]{\textbf{#1}}
\renewcommand{\bibnumfmt}[1]{\textbf{#1.}}

\titleformat{\section}{\normalfont\Large\bfseries\centering}{\thesection.}{0.5em}{}
\titleformat{\subsection}{\normalfont\bfseries}{\thesubsection.}{0.5em}{}

\newenvironment{myabstract}{\vspace{1em}\begin{adjustwidth}{3em}{3em}\begin{small}\textbf{Abstract.}}{\end{small}\end{adjustwidth}\vspace{1em}}
\newenvironment{mykeywords}{\vspace{1em}\begin{adjustwidth}{3em}{3em}\begin{small}\textbf{Keywords.}}{\end{small}\end{adjustwidth}\vspace{1em}}

\DeclareCaptionLabelSeparator{custom}{.}
\DeclareCaptionLabelFormat{custom}
{%
  \textsc{#1 #2}
}
\DeclareCaptionFormat{custom}
{%
  #1#2 #3
}
\captionsetup
{
  format=custom,%
  labelformat=custom,%
  labelsep=custom
}

\begin{document}

\begin{center}
  {\Large\bfseries Math 342 Tutorial} \\
  {\normalsize\bf July 30, 2025}
\end{center}

\noindent{\bf Question 1.}
\begin{enumerate}[{\bf (a)}]
\item If the two most common letters in a long ciphertext, encrypted by an
  affine transformation $C \equiv aP+b \pmod{26}$, are $X$ then $Q$, then what
  are the most likely values for $a$ and $b$.
\item If the two most common letters in a long ciphertext, encrypted by an
  affine transformation $C \equiv aP+b \pmod{26}$, are $W$ then $B$, then what
  are the most likely values for $a$ and $b$.
\end{enumerate}

\noindent{\bf Question 2.}
\begin{enumerate}[{\bf (a)}]
\item Decrypt MJMZK CXUNM GWIRY VCPUW MPRRW GMIOP MSNYS RYRAZ PXMCD WPRYE YXD
  which was encrypted using an affine transformation.
\item Decrypt WEZBF TBBNJ THNBT ADZQE TGTYR BZAJN ANOOZ ATWGN ABOVG FNWZV A
  which was encrypted using an affine transformation.
\item Decrypt PJXFJ SWJNX JMRTJ FVSUJ OOKWE OVAJR WHEOF JRWJO DJFFZ BJF
  which was encrypted using an affine transformation.
\end{enumerate}

\noindent{\bf Question 3.} What is the plaintext message that corresponds to the
ciphertext 1213 0902 0539 1208 1234 1103 1374 that is produced using modular
exponentiation with modulus $p=2591$ and encryption key $e=13$? \\

\noindent{\bf Question 4.} Show that the encryption and decryption procedures
are identical when encryption is done using modular exponentiation with modulus
$p=31$ and enciphering key $e=11$. \\

\noindent{\bf Question 5.} Suppose a cryptanalyst discovers a message $P$ that
is relatively prime to the enciphering modulus $n=pq$ used in the RSA cipher.
Show the cryptanalyst can factor $n$. \\

\noindent{\bf Question 6.} Show that it is extremely unlikely that a message
such as that described in the previous exercise can be discovered. Do this by
demonstrating that the probability that a message $P$ is not relatively prime to
$n$ is $\frac{1}{p}+\frac{1}{q}-\frac{1}{pq}$. Assume that it is equally likely
for a message to fall into each residue class modulo $n$. \\

\end{document}

