\documentclass[a4paper,11pt]{article}

\usepackage[utf8]{inputenc}
\usepackage[english]{babel}
\usepackage{amssymb, amsmath, amsthm, mathrsfs}
\usepackage[left=1.0in,right=1.0in,top=1.0in,bottom=1.0in]{geometry}
\usepackage[T1]{fontenc}
\usepackage{array}
\usepackage{longtable}
\usepackage{multirow}
\usepackage{calc}
\usepackage[inline,shortlabels]{enumitem}
\usepackage{changepage}
\usepackage{booktabs}
\usepackage{capt-of}
\usepackage{subcaption}
\usepackage[leftcaption]{sidecap}
\usepackage[numbers]{natbib}
\usepackage{times}
\usepackage{titlesec}
\usepackage{xcolor}
\usepackage{lineno}
\usepackage{xpatch}
\xpatchcmd\swappedhead{~}{.~}{}{}
\allowdisplaybreaks

\newtheoremstyle{mythm}
{}                % Space above
{}                % Space below
{\itshape}        % Theorem body font % (default is "\upshape")
{1.5em}                % Indent amount
{\scshape}       % Theorem head font % (default is \mdseries)
{.}               % Punctuation after theorem head % default: no punctuation
{0.5em}               % Space after theorem head
{}                % Theorem head spec
\theoremstyle{mythm}


\newtheorem*{theorem*}{Theorem}
\newtheorem{theorem}{Theorem}
\newtheorem{fact}[theorem]{Fact}
\newtheorem{proposition}[theorem]{Proposition}
\newtheorem{lemma}[theorem]{Lemma}
\newtheorem{corollary}[theorem]{Corollary}
\newtheorem{question}[theorem]{Question}
\newtheorem{result}[theorem]{Result}
\newtheorem{observation}[theorem]{Observation}
\newtheorem{conjecture}[theorem]{Conjecture}

\newtheoremstyle{mydef}
{}                % Space above
{}                % Space below
{}        % Theorem body font % (default is "\upshape")
{1.5em}                % Indent amount
{\scshape}       % Theorem head font % (default is \mdseries)
{.}               % Punctuation after theorem head % default: no punctuation
{0.5em}               % Space after theorem head
{}                % Theorem head spec
\theoremstyle{mydef}

\newtheorem{example}[theorem]{Example}
\newtheorem{definition}[theorem]{Definition}
\newtheorem{remark}[theorem]{Remark}
\newtheorem*{remark*}{Remark}

\makeatletter
\renewenvironment{proof}[1][\proofname]{\par
  \pushQED{\qed}%
  \normalfont \topsep6\p@\@plus6\p@\relax
  \trivlist
\item\relax
  {\hspace{1.5em}\itshape
    #1\@addpunct{.}}\hspace\labelsep\ignorespaces
}{%
  \popQED\endtrivlist\@endpefalse
}
\makeatother

\def\Box{\hskip1ex\vbox{\hrule height0.6pt\hbox{%
      \vrule height1.3ex width0.6pt\hskip0.8ex
      \vrule width0.6pt}\hrule height0.6pt
  }}
\renewcommand{\qed}{\Box}

\newcommand{\red}[1]{\textcolor{red}{#1}}
\newcommand{\blue}[1]{\textcolor{blue}{#1}}
\newcommand{\purple}[1]{\textcolor{magenta}{#1}}
\newcommand{\ddet}{\text{det}}
\renewcommand{\pmod}[1]{\text{ (mod $#1$)}}
\newcommand{\mmod}[2]{#1\text{ mod }#2}
\newcommand{\abs}[1]{\left\vert #1 \right\vert}
\newcommand{\C}{\mathbf{C}}
\newcommand{\Z}{\mathbf{Z}}
\newcommand{\N}{\mathbf{N}}
\newcommand{\LL}{\mathscr{G}}
\newcommand{\z}{\mathbin{\ooalign{$\hidewidth i \hidewidth$\cr$\phantom{+}$}}}
\newcommand{\y}{\mathbin{\ooalign{$\hidewidth j \hidewidth$\cr$\phantom{+}$}}}
\newcommand{\gf}{\text{GF}}
\newcommand{\ord}{\text{ord}}

\newcolumntype{R}{>{\scriptsize}r}
\newcolumntype{L}{>{\scriptsize}l}
\newcolumntype{C}{>{\scriptsize}c}

\renewcommand{\citenumfont}[1]{\textbf{#1}}
\renewcommand{\bibnumfmt}[1]{\textbf{#1.}}

\titleformat{\section}{\normalfont\Large\bfseries\centering}{\thesection.}{0.5em}{}
\titleformat{\subsection}{\normalfont\bfseries}{\thesubsection.}{0.5em}{}

\newenvironment{myabstract}{\vspace{1em}\begin{adjustwidth}{3em}{3em}\begin{small}\textbf{Abstract.}}{\end{small}\end{adjustwidth}\vspace{1em}}
\newenvironment{mykeywords}{\vspace{1em}\begin{adjustwidth}{3em}{3em}\begin{small}\textbf{Keywords.}}{\end{small}\end{adjustwidth}\vspace{1em}}

\DeclareCaptionLabelSeparator{custom}{.}
\DeclareCaptionLabelFormat{custom}
{%
  \textsc{#1 #2}
}
\DeclareCaptionFormat{custom}
{%
  #1#2 #3
}
\captionsetup
{
  format=custom,%
  labelformat=custom,%
  labelsep=custom
}

\begin{document}

\begin{center}
  {\Large\bfseries Math 342 Tutorial} \\
  {\normalsize\bf July 2, 2025}
\end{center}

Recall a group $G$ is cyclic if there is some $g \in G$ for which $G = \{g^n : n
\in \Z\}$. If $G$ is finite with $n$ distinct elements, then $G =
\{1,\,g,\,g^2,\dots,\,g^{n-1}\}$. We often use the notation $G = \langle g
\rangle$ in this case. \\

\noindent{\bf Question 1.} Let $G = \langle g \rangle$ be a finite cyclic group
of order $n$. Show that for every divisor $d$ of $n$, there is a unique subgroup
$H$ of $G$ of order $d$. Show, moreover, that $H$ is cyclic. \\

\noindent{\bf Question 2.} Fill in the details of the following argument to show
that $(\Z/p\Z)^*$, the nonzero residues modulo $p$, form a cyclic group.
\begin{enumerate}[{\bf (a)}]
\item Let $h = q_1^{r_1} \cdots q_s^{r_s}$ be the prime power
  factorization of $h = p-1$. Show that for every $1 \leqq i \leqq s$, there is
  a nonzero residue which is not a root of $x^{h/p_i}-1$ modulo $p$.
\item Let $a_i$ be a nonzero residue which is not a root of $x^{h/p_i}-1$, and
  define $b_i=a_i^{h/p_i^{r_i}}$. Show that $\ord_p(b_i)=p_i^{r_i}$.
\item Show the element $b = b_1 \cdots b_s$ has multiplicative order $h = p-1$.
  In particular, this shows that $(\Z/p\Z)^*$ is cyclic. We call $b$ a primitive
  root modulo $p$.
\end{enumerate}

Compare the previous question with Question 6 of the previous tutorial set. \\

\noindent{\bf Question 3.} Use Questions 1 and 2 to show the following.
\begin{enumerate*}[{\bf (a)}]
\item $a^p \equiv a \pmod{p}$ for every integer $a$.
\item $\left( \frac{a}{p} \right) \equiv a^{(p-1)/2} \pmod{p}$.
\end{enumerate*} \\

\noindent{\bf Question 4.} We will give a second proof of the fact that $\left(
  \frac{2}{p} \right) = (-1)^{(p^2-1)/8}$. Fill in the following details.
\begin{enumerate}[{\bf (a)}]
\item Let $i$ be the principal root of $-1$. Show that $(1+i)^2=2i$. Use this to
  show that $(1+i)^p = (1+i)i^{(p-1)/2}2^{(p-1)/2}$.
\item Show that $\left( \frac{2}{p} \right)(1+i)i^{(p-1)/2} \equiv
  1+i(-1)^{(p-1)/2} \pmod{p}$. Use this to show that $\left( \frac{2}{p} \right)
  \equiv (-1)^{(p\pm1)/4} \pmod{p}$ predicated upon whether $\frac{p-1}{2}$ is
  even or odd.
\item Deduce that $\left( \frac{2}{p} \right) \equiv (-1)^{(p^2-1)/8}$.
\end{enumerate}

\noindent{\bf Question 5.} Use quadratic reciprocity to determine $\left(
  \frac{3}{p} \right)$. \\

\noindent{\bf Question 6.} Euler's Theorem is the following. {\it Let $p$ be an odd
prime, and let $a$ be an integer not divisible by $p$. If $q$ is a prime with $p
\equiv \pm q \pmod{4a}$, then $\left( \frac{a}{p} \right) = \left( \frac{a}{p}
\right)$.} Show that Euler's Theorem is equivalent to the law of quadratic
reciprocity shown in class. [Hint: For necessity, it is convenient to consider
the cases $a=2$, $a$ an odd prime, and $a$ composite separately.] \\

\end{document}

