\documentclass[a4paper,11pt]{article}

\usepackage[utf8]{inputenc}
\usepackage[english]{babel}
\usepackage{amssymb, amsmath, amsthm, mathrsfs}
\usepackage[left=1.0in,right=1.0in,top=1.0in,bottom=1.0in]{geometry}
\usepackage[T1]{fontenc}
\usepackage{array}
\usepackage{longtable}
\usepackage{multirow}
\usepackage{calc}
\usepackage[inline,shortlabels]{enumitem}
\usepackage{changepage}
\usepackage{booktabs}
\usepackage{capt-of}
\usepackage{subcaption}
\usepackage[leftcaption]{sidecap}
\usepackage[numbers]{natbib}
\usepackage{times}
\usepackage{titlesec}
\usepackage{xcolor}
\usepackage{lineno}
\usepackage{xpatch}
\xpatchcmd\swappedhead{~}{.~}{}{}
\allowdisplaybreaks

\newtheoremstyle{mythm}
{}                % Space above
{}                % Space below
{\itshape}        % Theorem body font % (default is "\upshape")
{1.5em}                % Indent amount
{\scshape}       % Theorem head font % (default is \mdseries)
{.}               % Punctuation after theorem head % default: no punctuation
{0.5em}               % Space after theorem head
{}                % Theorem head spec
\theoremstyle{mythm}


\newtheorem*{theorem*}{Theorem}
\newtheorem{theorem}{Theorem}
\newtheorem{fact}[theorem]{Fact}
\newtheorem{proposition}[theorem]{Proposition}
\newtheorem{lemma}[theorem]{Lemma}
\newtheorem{corollary}[theorem]{Corollary}
\newtheorem{question}[theorem]{Question}
\newtheorem{result}[theorem]{Result}
\newtheorem{observation}[theorem]{Observation}
\newtheorem{conjecture}[theorem]{Conjecture}

\newtheoremstyle{mydef}
{}                % Space above
{}                % Space below
{}        % Theorem body font % (default is "\upshape")
{1.5em}                % Indent amount
{\scshape}       % Theorem head font % (default is \mdseries)
{.}               % Punctuation after theorem head % default: no punctuation
{0.5em}               % Space after theorem head
{}                % Theorem head spec
\theoremstyle{mydef}

\newtheorem{example}[theorem]{Example}
\newtheorem{definition}[theorem]{Definition}
\newtheorem{remark}[theorem]{Remark}
\newtheorem*{remark*}{Remark}

\makeatletter
\renewenvironment{proof}[1][\proofname]{\par
  \pushQED{\qed}%
  \normalfont \topsep6\p@\@plus6\p@\relax
  \trivlist
\item\relax
  {\hspace{1.5em}\itshape
    #1\@addpunct{.}}\hspace\labelsep\ignorespaces
}{%
  \popQED\endtrivlist\@endpefalse
}
\makeatother

\def\Box{\hskip1ex\vbox{\hrule height0.6pt\hbox{%
      \vrule height1.3ex width0.6pt\hskip0.8ex
      \vrule width0.6pt}\hrule height0.6pt
  }}
\renewcommand{\qed}{\Box}

\newcommand{\red}[1]{\textcolor{red}{#1}}
\newcommand{\blue}[1]{\textcolor{blue}{#1}}
\newcommand{\purple}[1]{\textcolor{magenta}{#1}}
\newcommand{\ddet}{\text{det}}
\renewcommand{\pmod}[1]{\text{ (mod $#1$)}}
\newcommand{\mmod}[2]{#1\text{ mod }#2}
\newcommand{\abs}[1]{\left\vert #1 \right\vert}
\newcommand{\C}{\mathbf{C}}
\newcommand{\Z}{\mathbf{Z}}
\newcommand{\N}{\mathbf{N}}
\newcommand{\LL}{\mathscr{G}}
\newcommand{\z}{\mathbin{\ooalign{$\hidewidth i \hidewidth$\cr$\phantom{+}$}}}
\newcommand{\y}{\mathbin{\ooalign{$\hidewidth j \hidewidth$\cr$\phantom{+}$}}}
\newcommand{\gf}{\text{GF}}

\newcolumntype{R}{>{\scriptsize}r}
\newcolumntype{L}{>{\scriptsize}l}
\newcolumntype{C}{>{\scriptsize}c}

\renewcommand{\citenumfont}[1]{\textbf{#1}}
\renewcommand{\bibnumfmt}[1]{\textbf{#1.}}

\titleformat{\section}{\normalfont\Large\bfseries\centering}{\thesection.}{0.5em}{}
\titleformat{\subsection}{\normalfont\bfseries}{\thesubsection.}{0.5em}{}

\newenvironment{myabstract}{\vspace{1em}\begin{adjustwidth}{3em}{3em}\begin{small}\textbf{Abstract.}}{\end{small}\end{adjustwidth}\vspace{1em}}
\newenvironment{mykeywords}{\vspace{1em}\begin{adjustwidth}{3em}{3em}\begin{small}\textbf{Keywords.}}{\end{small}\end{adjustwidth}\vspace{1em}}

\DeclareCaptionLabelSeparator{custom}{.}
\DeclareCaptionLabelFormat{custom}
{%
  \textsc{#1 #2}
}
\DeclareCaptionFormat{custom}
{%
  #1#2 #3
}
\captionsetup
{
  format=custom,%
  labelformat=custom,%
  labelsep=custom
}

\begin{document}

\begin{center}
  {\Large\bfseries Math 342 Tutorial} \\
  {\normalsize\bf June 11, 2025}
\end{center}

\noindent{\bf Question 1.} Find all solutions to following systems of
congruences in two ways: first, using the Chinese Remainder Theorem; and second,
by iteratively solving and substituting linear congruences.

\begin{enumerate}[(a)]
\item $x \equiv 1 \pmod{2},\, x \equiv 2 \pmod{3},\, x \equiv 3 \pmod{5}$.
\item $x \equiv 0 \pmod{2},\, x \equiv 0 \pmod{3},\, x \equiv 1 \pmod{5},\, x
  \equiv 6 \pmod{7}$.
\end{enumerate}

\blue{
  \begin{enumerate}[(a)]
  \item We first employ CRT. We have $M=2 \cdot 3 \cdot 5 = 30$, $M_1=15$,
    $M_2=10$, $M_3=6$. The inverses of $M_1,\,M_2,\,M_3$ modulo 2, 3, 5 are all
    1. Then the unique solution modulo $M$ is given by
    \[
      (1)(15)(1)+(2)(10)(1)+(3)(6)(1) = 15+20+18 = 53 \equiv 23 \pmod{30}.
    \]
    Now we use the second method. We are given $x=1+2t$, so $1+2t \equiv 2
    \pmod{3}$. Solving, we obtain $t \equiv 2 \pmod{3}$ so that $t=2+3s$. Then
    $x = 1+2(2+3s) = 5+6s$. Then $5+6s \equiv 3 \pmod{5}$. Solving again, we
    have $s \equiv 3 \pmod{5}$ so that $s=3+5r$. Then $x=5+6(3+5r) = 23+30r
    \equiv 23 \pmod{30}$, and we're done.
  \item Note that we see at once by inspection that 6 is the required solution.
    However, we carry through the process using CRT. We have that $M=2 \cdot 3
    \cdot 5 \cdot 7 = 210$, $M_3=42$, $M_4=30$. The inverses of $M_3$ and $M_4$
    modulo 5, 7 are 3, 4, respectively. Then
    \[
      x=(42)(3) + (6)(30)(4) = 846 \equiv 6 \pmod{210}.
    \]
    We now use the second method. The first two congruences imply $x=6t$. Then
    $6t \equiv 1 \pmod{5}$, hence $t \equiv 1 \pmod{5}$ and $t=1+5s$. Then
    $x = 6(1+5s) = 6+30s$. Next, $6+30s \equiv 6 \pmod{7}$. Solving, we have $s
    \equiv 0 \pmod{7}$ and $s=7r$. Then $x = 6+30(7r) = 6+210r \equiv 6
    \pmod{210}$, and we're done. \\
  \end{enumerate}
}

\noindent{\bf Question 2.} Give the following generalization of the Chinese
Remainder Theorem. Let $m_1,\dots,\,m_r$ be pairwise coprime integers. Then the
system $a_1x \equiv b_1 \pmod{m_1}, \dots,\, a_rx \equiv b_r \pmod{m_r}$ has
exactly one solution modulo
$\frac{m_1}{(a_1,\,m_1)}\cdots\frac{m_r}{(a_r,\,m_r)}$ if and only if each
$(a_i,\,m_i) \mid b_i$. \\

\blue{Note that $a_ix \equiv b_i \pmod{m_i}$ is soluble if and only if
  $(a_i,\,m_i) \mid b_i$. In this case, $a_ix \equiv b_i \pmod{m_i}$ is
  equivalent to $x \equiv b_i/(a_i,\,m_i) \pmod{m_i/(a_i,\,m_i)}$. The rest is
  simply the usual CTR since the $\{m_i/(a_i,\,m_i)\}$ are pairwise coprime.} \\

\noindent{\bf Question 3.}
\begin{enumerate*}[(a)]
\item Show that the system of congruences $x \equiv a_1 \pmod{m_1}, \dots,\, x
  \equiv a_r \pmod{m_r}$ has a solution if and only if $(m_i,\,m_j) \mid
  (a_i-a_j)$ for all $i<j$. Show that if a solution exists, then it is unique
  modulo $[m_1,\dots,\,m_r]$. [Hint: succesively substitute linear equations.]
\item Solve the system $x \equiv 4 \pmod{6},\, x \equiv 13 \pmod{15}$.
\item Solve the system $x \equiv 5 \pmod{6},\, x \equiv 3 \pmod{10},\, x \equiv
  8 \pmod{15}$.
\item Does the system $x \equiv 1 \pmod{8},\, x \equiv 3 \pmod{9}$, $x \equiv 2
  \pmod{12}$ have any solutions?
\end{enumerate*}

\blue{
  \begin{enumerate}[(a)]
  \item The proof is by induction on $r$. Consider the case $r=2$, i.e., an
    arbitrary system of 2 linear congruences $x \equiv a_1 \pmod{m_1}$ and $x
    \equiv a_2 \pmod{m_2}$. The first congruence implies $x=a_1+m_1k$ for some
    $k \in \Z$. Substituting, we have $a_1+m_1k \equiv a_2 \pmod{m_2}$, i.e.,
    $m_1k \equiv a_2-a_1 \pmod{m_2}$. This has a solution in $k$ if and only if
    $(m_1,\,m_2) \mid a_2-a_1$. Assume $k_0$ is such a solution; then all
    incongruent solutions modulo $m_2$ are given by $k =
    k_0+\frac{m_2}{(m_1,\,m_2)}t$. Then
    \[
      x = a_1 + m_1\left( k_0+\frac{m_2}{(m_1,\,m_2)}t \right) =
      a_1+k_0m_1+[m_1,\,m_2]t.
    \]
    Therefore, the solution $x_0=a_1+k_0m_1$ is unique modulo $[m_1,\,m_2]$.
    Since the system was arbitrary, we have shown the base case. \\
    Next let $r > 2$ be arbitrary, and suppose the result holds for $r-1$. If
    there is such a solution to the system $x \equiv a_i \pmod{m_i}$,
    $i=1, \dots,\, r$, then in particular there is a solution to the system $x
    \equiv a_i \pmod{m_i}$, $x \equiv a_r \pmod{m_r}$ for each $i =1, \dots,\,
    r-1$. From part (a), this implies that $(m_i,\,m_r) \mid a_r-a_i$, $i=1
    ,\dots,\, r-1$. From the inductive hypothesis, we also have $(m_i,\,m_j)
    \mid a_j-a_i$ for $1 \leqq i < j < r$. We therefore have necessity for the
    given $r$. \\
    Next, suppose $(m_i,\,m_j) \mid a_j-a_i$ for each $1 \leqq i < j
    \leqq r$. In particular, by the inductive hypothesis, there is a unique
    solution to the system $x \equiv a_i \pmod{m_i}$, $i=1,\dots,\,r-1$ modulo
    $M=[m_1,\dots,\,m_{r-1}]$, say $A \pmod{M}$. We next consider the system $x
    \equiv A \pmod{M}$, $x \equiv a_r \pmod{m_r}$. From the base case, this
    admits a solution if and only if $(M,\,m_r) \mid A-a_r$. We are given that
    $(m_i,\,m_r) \mid a_i-a_r$ and $(m_i,\,m_r) \mid m_i \mid a_i-A$ for each
    $i = 1,\dots,\, r-1$. Hence, $(m_i,\,m_r) \mid (a_i-a_r)-(a_i-A) = A-a_r$.
    Since this holds for each $i < r$, we have that
    $[(m_1,\,m_r),\dots,\,(m_{r-1},\,m_r)] \mid A-a_r$. But
    $[(m_1,\,m_r),\dots,\,(m_{r-1},\,m_r)] =
    ([m_1,\dots,\,m_{r-1}],\,m_r)=(M,\,m_r)$. In other words, $(M,\,m_r) \mid
    A-a_r$, as required. \\
    Because the system was arbitrary, the result holds for this $r$. By
    mathematical induction, the result holds for all $r \geqq 2$.
  \item Note $(6,\,15)=3 \mid 13-4 = 9$, so there is indeed a solution. From
    part (a), we desire a solution $k$ to $15k \equiv 4-13  \equiv 3 \pmod{6}$.
    We may simply take $k=1$. Then $x \equiv 13+15 \equiv 28 \pmod{30}$.
  \item One may check that the system is consistent, i.e., that $(m_i,\,m_j)
    \mid a_i-a_j$ for each $i<j$. We first find a solution to $x \equiv 5
    \pmod{6}$, $x \equiv 3 \pmod{10}$. So, we seek a solution to $10k \equiv 5-3
    \pmod{6}$ which is equivalent to $4k \equiv 2 \pmod{6}$. Taking $k=2$, we
    see that $A=3+2(10)=23$ is to unique solution modulo $[6,\,10]=30$. \\
    Now we seek a solution to $x \equiv 23 \pmod{30}$, $x \equiv 8 \pmod{15}$.
    But $23 \equiv 8 \pmod{15}$, so we're done.
  \item There is no solution because $(12,\,8)=4$ does not divide $2-1=1$. \\
  \end{enumerate}
}

\noindent{\bf Question 4.} Show there are arbitrarily long strings of
consecutive integers each divisible by a perfect square greater than 1. [Hint:
Use CRT to show there is a simultaneous solution to the system $x \equiv 0
\pmod{4}$, $x \equiv -1 \pmod{9}$, $x \equiv -2 \pmod{25}$, $\dots$, $x \equiv
-k+1 \pmod{p_k^2}$ where $p_k$ is the $k$th prime.] \\

\blue{
  Following the hint, we consider the system
  \[
    x \equiv 0 \pmod{4},\, x \equiv -1 \pmod{9}, \dots,\, x \equiv -k+1
    \pmod{p_k^2}.
  \]
  By CRT, there is a unique solution $N$ modulo $4p_2^2 \cdots p_k^2$. Now the
  integers in the sequence $N,\,N+1,\dots,\,N+k-1$ are each divisible by a
  square because $p_j^2 \mid N+j-1$ as $N \equiv -j+1 \pmod{p_j^2}$ as the
  solution to the CRT problem.
} \\

\noindent{\bf Question 5.} Let $m=2^{e_0}p_1^{e_1} \cdots p_k^{e_k}$. Show the
congruence $x^2 \equiv 1 \pmod{m}$ has exactly $r^{r+s}$ solutions where $s=a_0$
if $0 \leqq a_0 \leqq 2$, and $s=4$ for $a_0>2$. [Hint: Use question 12 from the
May 28th tutorial set.] \\

\blue{The congruence $x^2 \equiv 1 \pmod{m}$ is equivalent to the system $x^2
  \equiv 1 \pmod{2^{e_0}}$, $x^2 \equiv 1 \pmod{p_i^{e_i}}$, $i=1,\dots,\,r$.
  Each of the odd prime congruences has two solutions given by $\pm 1
  \pmod{p_i^{e_i}}$. For $a_0=0$, there is nothing to report. For $a_0=1$, there
  is one solution to $x^2 \equiv 1 \pmod{2}$. For $a_0=2$, there are two
  solutions given by $x = \pm 1 \pmod{4}$. For $a_0>2$, there are four solutions
  given by $x = \pm 1\text{ or }\pm(1+2^{k-1}) \pmod{2^k}$. In any event, there
  are $2^{s+r}$ solutions to $x^2 \equiv 1 \pmod{m}$.} \\

\noindent{\bf Question 6.} Find all solutions to the following congruences.
\begin{enumerate*}[(a)]
\item $x^3+8x^2-x-1 \equiv 0 \pmod{121}$.
\item $x^2+4x+2 \equiv 0 \pmod{343}$.
\item $13x^7-42x-649 \equiv 0 \pmod{1323}$.
\end{enumerate*}

\blue{
  \begin{enumerate}[(a)]
  \item Note that $121=11^2$, so we apply Hensel's Lemma. We first solve
    $f(x)=x^3+8x^2-x-1 \equiv 0 \pmod{11}$. Checking all possibilities, we find
    that $x=4,\,5$. Next, the derivative is $f'(x)=3x^2+16x-1$. Observe that 4
    is not a root of the derivative modulo 11 but 5 is. We can lift 4 to a root
    $r$ of $f(x) \pmod{121}$ as $r=4+11t$ where
    \[
      t \equiv -\overline{f'(4)}\left( \frac{f(4)}{11} \right) \equiv
      -\overline{111}\left( \frac{187}{11} \right) \equiv 5 \pmod{11}.
    \]
    Therefore, $r=59$. For the root 5, oberseve that $f(5) \not\equiv 0
    \pmod{121}$, hence there are no liftings to roots of $f(x) \pmod{121}$. We
    have shown that the only root of $f(x) \pmod{121}$ is 59.
  \item We have $343=7^3$, so we apply Hensel's Lemma. Let $f(x)=x^2+4x+2$. The
    solutions of $f(x) \equiv 0 \pmod{7}$ are given by $x=1,\,2$. The derivative
    is $f'(x)=2x+4$. Note that neither $f(1)$ nor $f(2)$ is 0 modulo 49. We can
    therefore lift them to unique roots of $f(x)$ modulo 49. Using the lemma
    these are given by 8 and 37, respectively. Again, neither $f'(8)$ nor
    $f'(49)$ are zero modulo 7, hence they can be lifted uniquely. These lifts
    are given by 106 and 233, respectively.
  \item Note $1323=3^37^2$. Let $f(x)=13x^7-42x-649$. We handle the
    characteristic 3 first. The only solution to $f(x) \equiv 0 \pmod{3}$ is
    $x=1$. Observe that $f'(1) \neq 0 \pmod{3}$. Therefore, we may lift $x=1$ to
    a unique root of $f(x)$ modulo $3^3$. This root is given by 22. \\
    Next, we handle the characteristic 7. Again, there is a unique root $f(x)$
    modulo 7. It is given by $x=2$. Now $f'(2) \equiv 0 \pmod{7}$ and $f(2)
    \equiv 0 \pmod{49}$, hence it has 7 lifts given by $2+7t$ for $0 \leqq t <
    7$, i.e., 2, 9, 16, 23, 30, 37, and 44. \\
    Next, we need to pair each solution for 27 with each solution for 49. The
    system $x \equiv 22 \pmod{27}$, $x \equiv 2 \pmod{49}$ has solution $x
    \equiv 1129 \pmod{1323}$. The remaining systems have solutions 940, 751,
    562, 373, 184, 1318.
  \end{enumerate}
}

\noindent{\bf Question 7.} Suppose $(a,\,p)=1$. Use Hensel's Lemma to find a
recursive formula for the solutions of $ax \equiv 1 \pmod{p^k}$ for all positive
integers $k$. \\

\blue{Since $(a,\,p)=1$, $a$ has an inverse $b$ modulo $p$. Define $f(x)=ax-1$.
  Then $f(b)=0 \pmod{p}$ is the unique solution modulo $p$. But $f'(a)=a
  \not\equiv 0 \pmod{p}$, so we can lift it uniquely to larger characteristics.
  We have $r_k = r_{k-1}-f(r_{k-1})\overline{f'(b)} = r_{k-1}-(ar_{k-1}-1)b =
  r_{k-1}(1-ab)+b$.}

\end{document}

