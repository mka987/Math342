\documentclass[a4paper,11pt]{article}

\usepackage[utf8]{inputenc}
\usepackage[english]{babel}
\usepackage{amssymb, amsmath, amsthm, mathrsfs}
\usepackage[left=1.0in,right=1.0in,top=1.0in,bottom=1.0in]{geometry}
\usepackage[T1]{fontenc}
\usepackage{array}
\usepackage{longtable}
\usepackage{multirow}
\usepackage{calc}
\usepackage[inline,shortlabels]{enumitem}
\usepackage{changepage}
\usepackage{booktabs}
\usepackage{capt-of}
\usepackage{subcaption}
\usepackage[leftcaption]{sidecap}
\usepackage[numbers]{natbib}
\usepackage{times}
\usepackage{titlesec}
\usepackage{xcolor}
\usepackage{lineno}
\usepackage{xpatch}
\xpatchcmd\swappedhead{~}{.~}{}{}
\allowdisplaybreaks

\newtheoremstyle{mythm}
{}                % Space above
{}                % Space below
{\itshape}        % Theorem body font % (default is "\upshape")
{1.5em}                % Indent amount
{\scshape}       % Theorem head font % (default is \mdseries)
{.}               % Punctuation after theorem head % default: no punctuation
{0.5em}               % Space after theorem head
{}                % Theorem head spec
\theoremstyle{mythm}


\newtheorem*{theorem*}{Theorem}
\newtheorem{theorem}{Theorem}
\newtheorem{fact}[theorem]{Fact}
\newtheorem{proposition}[theorem]{Proposition}
\newtheorem{lemma}[theorem]{Lemma}
\newtheorem{corollary}[theorem]{Corollary}
\newtheorem{question}[theorem]{Question}
\newtheorem{result}[theorem]{Result}
\newtheorem{observation}[theorem]{Observation}
\newtheorem{conjecture}[theorem]{Conjecture}

\newtheoremstyle{mydef}
{}                % Space above
{}                % Space below
{}        % Theorem body font % (default is "\upshape")
{1.5em}                % Indent amount
{\scshape}       % Theorem head font % (default is \mdseries)
{.}               % Punctuation after theorem head % default: no punctuation
{0.5em}               % Space after theorem head
{}                % Theorem head spec
\theoremstyle{mydef}

\newtheorem{example}[theorem]{Example}
\newtheorem{definition}[theorem]{Definition}
\newtheorem{remark}[theorem]{Remark}
\newtheorem*{remark*}{Remark}

\makeatletter
\renewenvironment{proof}[1][\proofname]{\par
  \pushQED{\qed}%
  \normalfont \topsep6\p@\@plus6\p@\relax
  \trivlist
\item\relax
  {\hspace{1.5em}\itshape
    #1\@addpunct{.}}\hspace\labelsep\ignorespaces
}{%
  \popQED\endtrivlist\@endpefalse
}
\makeatother

\def\Box{\hskip1ex\vbox{\hrule height0.6pt\hbox{%
      \vrule height1.3ex width0.6pt\hskip0.8ex
      \vrule width0.6pt}\hrule height0.6pt
  }}
\renewcommand{\qed}{\Box}

\newcommand{\red}[1]{\textcolor{red}{#1}}
\newcommand{\blue}[1]{\textcolor{blue}{#1}}
\newcommand{\purple}[1]{\textcolor{magenta}{#1}}
\newcommand{\ddet}{\text{det}}
\renewcommand{\pmod}[1]{\text{ (mod $#1$)}}
\newcommand{\mmod}[2]{#1\text{ mod }#2}
\newcommand{\abs}[1]{\left\vert #1 \right\vert}
\newcommand{\C}{\mathbf{C}}
\newcommand{\Z}{\mathbf{Z}}
\newcommand{\N}{\mathbf{N}}
\newcommand{\LL}{\mathscr{G}}
\newcommand{\z}{\mathbin{\ooalign{$\hidewidth i \hidewidth$\cr$\phantom{+}$}}}
\newcommand{\y}{\mathbin{\ooalign{$\hidewidth j \hidewidth$\cr$\phantom{+}$}}}
\newcommand{\gf}{\text{GF}}
\newcommand{\ord}{\text{ord}}

\newcolumntype{R}{>{\scriptsize}r}
\newcolumntype{L}{>{\scriptsize}l}
\newcolumntype{C}{>{\scriptsize}c}

\renewcommand{\citenumfont}[1]{\textbf{#1}}
\renewcommand{\bibnumfmt}[1]{\textbf{#1.}}

\titleformat{\section}{\normalfont\Large\bfseries\centering}{\thesection.}{0.5em}{}
\titleformat{\subsection}{\normalfont\bfseries}{\thesubsection.}{0.5em}{}

\newenvironment{myabstract}{\vspace{1em}\begin{adjustwidth}{3em}{3em}\begin{small}\textbf{Abstract.}}{\end{small}\end{adjustwidth}\vspace{1em}}
\newenvironment{mykeywords}{\vspace{1em}\begin{adjustwidth}{3em}{3em}\begin{small}\textbf{Keywords.}}{\end{small}\end{adjustwidth}\vspace{1em}}

\DeclareCaptionLabelSeparator{custom}{.}
\DeclareCaptionLabelFormat{custom}
{%
  \textsc{#1 #2}
}
\DeclareCaptionFormat{custom}
{%
  #1#2 #3
}
\captionsetup
{
  format=custom,%
  labelformat=custom,%
  labelsep=custom
}

\begin{document}

\begin{center}
  {\Large\bfseries Math 342 Tutorial} \\
  {\normalsize\bf June 25, 2025}
\end{center}

\noindent{\bf Question 1.} Show that if $\overline{a}$ is an inverse of $a$
modulo $n$, then $\ord_n\overline{a} = \ord_na$. \\

\noindent{\bf Question 2.}  Assume that $(a,\,n)=1=(b,\,n)$. Do the following.
\begin{enumerate*}[{\bf (a)}]
\item If $(\ord_na=\ord_nb) = 1$, then $\ord_nab = \ord_na \cdot \ord_nb$.
\item If we do not assume $(\ord_na=\ord_nb) = 1$, then what can be said about
  $\ord_nab$.
\end{enumerate*} \\

\noindent{\bf Question 3.}
\begin{enumerate*}[{\bf (a)}]
\item Suppose $d \mid \phi(n)$. Is it true that there is an integer $a$ for
  which $\ord_na=d$.
\item Show that if $(a,\,n)=1$ and $\ord_na=st$, then $\ord_na^t=s$.

\item Show that if $(a,\,n)=1$ and $\ord_na=n-1$, then $n$ is prime.
\end{enumerate*} \\

\noindent{\bf Question 4.} Show that $r$ is a primitive root modulo the prime
$p$ if and only if $r$ is an integer with $(r,\,p)=1$ such that
\[
  r^{(p-1)/q} \not\equiv 1 \pmod{p}
\]

\noindent for every prime divisor $q$ of $p-1$. \\

\noindent{\bf Question 5.} Let $m=a^n-1$. Show the following.
\begin{enumerate*}[{\bf (a)}]
\item $\ord_ma=n$, and
\item $n \mid \phi(m)$.
\end{enumerate*} \\

\noindent{\bf Question 6.} Let $p$ be a prime, and let $\phi(p-1)=q_1^{e_1}
\cdots q_k^{e_k}$ where each $q_i$ is prime.
\begin{enumerate*}[{\bf (a)}]
\item Show there are integers $a_1,\dots,\,a_k$ such that $\ord_pa_i=q_i^{e_i}$
  for each $i=1,\dots,\,k$.
\item Show that $a=a_1 \cdots a_k$ is a primitive root modulo $p$.
\item Follow the procedure outlined above to show find a primitive root modulo
  29.
\end{enumerate*} \\

\noindent{\bf Question 7.} Show that if $p$ is an odd prime then
\[
  \left( \frac{-2}{p} \right) =
  \begin{cases}
    1 & \text{if $p \equiv 1,\,3 \pmod{8}$,} \\
    -1 & \text{if $p \equiv -1,\,-3 \pmod{8}$.}
  \end{cases}
\]

\noindent{\bf Question 8.} Determine those primes $p$ for which $(-3 \mid p) =
-1$ and those primes $q$ for which $(-3 \mid q) = 1$. \\

\noindent{\bf Question 9.} Prove that 5 is a quadratic residue of an odd prime
$p$ if $p \equiv \pm1 \pmod{10}$, and that 5 is a non residue if $p \equiv \pm 3
\pmod{10}$. \\

\end{document}

