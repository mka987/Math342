\documentclass[a4paper,11pt]{article}

\usepackage[utf8]{inputenc}
\usepackage[english]{babel}
\usepackage{amssymb, amsmath, amsthm, mathrsfs}
\usepackage[left=1.0in,right=1.0in,top=1.0in,bottom=1.0in]{geometry}
\usepackage[T1]{fontenc}
\usepackage{array}
\usepackage{longtable}
\usepackage{multirow}
\usepackage{calc}
\usepackage[inline,shortlabels]{enumitem}
\usepackage{changepage}
\usepackage{booktabs}
\usepackage{capt-of}
\usepackage{subcaption}
\usepackage[leftcaption]{sidecap}
\usepackage[numbers]{natbib}
\usepackage{times}
\usepackage{titlesec}
\usepackage{xcolor}
\usepackage{lineno}
\usepackage{xpatch}
\xpatchcmd\swappedhead{~}{.~}{}{}
\allowdisplaybreaks

\newtheoremstyle{mythm}
{}                % Space above
{}                % Space below
{\itshape}        % Theorem body font % (default is "\upshape")
{1.5em}                % Indent amount
{\scshape}       % Theorem head font % (default is \mdseries)
{.}               % Punctuation after theorem head % default: no punctuation
{0.5em}               % Space after theorem head
{}                % Theorem head spec
\theoremstyle{mythm}


\newtheorem*{theorem*}{Theorem}
\newtheorem{theorem}{Theorem}
\newtheorem{fact}[theorem]{Fact}
\newtheorem{proposition}[theorem]{Proposition}
\newtheorem{lemma}[theorem]{Lemma}
\newtheorem{corollary}[theorem]{Corollary}
\newtheorem{question}[theorem]{Question}
\newtheorem{result}[theorem]{Result}
\newtheorem{observation}[theorem]{Observation}
\newtheorem{conjecture}[theorem]{Conjecture}

\newtheoremstyle{mydef}
{}                % Space above
{}                % Space below
{}        % Theorem body font % (default is "\upshape")
{1.5em}                % Indent amount
{\scshape}       % Theorem head font % (default is \mdseries)
{.}               % Punctuation after theorem head % default: no punctuation
{0.5em}               % Space after theorem head
{}                % Theorem head spec
\theoremstyle{mydef}

\newtheorem{example}[theorem]{Example}
\newtheorem{definition}[theorem]{Definition}
\newtheorem{remark}[theorem]{Remark}
\newtheorem*{remark*}{Remark}

\makeatletter
\renewenvironment{proof}[1][\proofname]{\par
  \pushQED{\qed}%
  \normalfont \topsep6\p@\@plus6\p@\relax
  \trivlist
\item\relax
  {\hspace{1.5em}\itshape
    #1\@addpunct{.}}\hspace\labelsep\ignorespaces
}{%
  \popQED\endtrivlist\@endpefalse
}
\makeatother

\def\Box{\hskip1ex\vbox{\hrule height0.6pt\hbox{%
      \vrule height1.3ex width0.6pt\hskip0.8ex
      \vrule width0.6pt}\hrule height0.6pt
  }}
\renewcommand{\qed}{\Box}

\newcommand{\red}[1]{\textcolor{red}{#1}}
\newcommand{\blue}[1]{\textcolor{blue}{#1}}
\newcommand{\purple}[1]{\textcolor{magenta}{#1}}
\newcommand{\ddet}{\text{det}}
\renewcommand{\pmod}[1]{\text{ (mod $#1$)}}
\newcommand{\mmod}[2]{#1\text{ mod }#2}
\newcommand{\abs}[1]{\left\vert #1 \right\vert}
\newcommand{\C}{\mathbf{C}}
\newcommand{\Z}{\mathbf{Z}}
\newcommand{\N}{\mathbf{N}}
\newcommand{\LL}{\mathscr{G}}
\newcommand{\z}{\mathbin{\ooalign{$\hidewidth i \hidewidth$\cr$\phantom{+}$}}}
\newcommand{\y}{\mathbin{\ooalign{$\hidewidth j \hidewidth$\cr$\phantom{+}$}}}
\newcommand{\gf}{\text{GF}}
\newcommand{\ord}{\text{ord}}

\newcolumntype{R}{>{\scriptsize}r}
\newcolumntype{L}{>{\scriptsize}l}
\newcolumntype{C}{>{\scriptsize}c}

\renewcommand{\citenumfont}[1]{\textbf{#1}}
\renewcommand{\bibnumfmt}[1]{\textbf{#1.}}

\titleformat{\section}{\normalfont\Large\bfseries\centering}{\thesection.}{0.5em}{}
\titleformat{\subsection}{\normalfont\bfseries}{\thesubsection.}{0.5em}{}

\newenvironment{myabstract}{\vspace{1em}\begin{adjustwidth}{3em}{3em}\begin{small}\textbf{Abstract.}}{\end{small}\end{adjustwidth}\vspace{1em}}
\newenvironment{mykeywords}{\vspace{1em}\begin{adjustwidth}{3em}{3em}\begin{small}\textbf{Keywords.}}{\end{small}\end{adjustwidth}\vspace{1em}}

\DeclareCaptionLabelSeparator{custom}{.}
\DeclareCaptionLabelFormat{custom}
{%
  \textsc{#1 #2}
}
\DeclareCaptionFormat{custom}
{%
  #1#2 #3
}
\captionsetup
{
  format=custom,%
  labelformat=custom,%
  labelsep=custom
}

\begin{document}

\begin{center}
  {\Large\bfseries Math 342 Tutorial} \\
  {\normalsize\bf July 23, 2025}
\end{center}

\noindent{\bf Fast Fourier Transform.} Let $n=2^m$, $a(x)=\sum_{j=0}^{n-1}a_jx^j$,
and $\omega=e^{2\pi\sqrt{-1}/n}$. The Fourier transform of of $a(x)$ is the
vector $(\hat a_0, \dots,\,\hat a_{n-1})$ where $\hat a_i = \sum_j
a_j\omega^{ij}$. Given $a(x) = \sum_{j=0}^{n-1}a_jx^j$, define
$A(x)=\sum_{j=0}^{2n-1}a_jx^j$. Show the following.
\begin{enumerate}[{\bf (a)}]
\item Given two polynomials $a$ and $b$ of degree $n=2^k$, show that the Fourier
  transform of $c(x)=a(x)b(x)$ is $(\hat c_0, \dots,\,\hat c_{2n-1})$ where
  $\hat c_i = \hat A_i \hat B_i$.
\item Show that $(a_0, \dots,\,a_{n-1}) \mapsto (\bar a_0,\dots,\,\bar a_{n-1})$
  where $\bar a_i = n^{-1}\sum_{j=0}^{n-1}a_i\omega^{-ij}$ is an inverse of the
  Fourier transform.
\item Show that we can use the Fourier transform to multiply two polynomials in
  $O(n\log n)$ time. [Hint: Decompose a polynomial $a(x)$ of degree $n=2^k$ as
  $a(x) = b(x^2)+xc(x^2)$ where $b(x)$ and $c(x)$ are the terms of even and odd
  exponent, respectively. Use the fact that $\{1,\omega,\dots,\omega^{n-1}\}$
  are symmetric in the sense that $\omega^{n/2+i}=-\omega^i$.]
\end{enumerate}

\noindent{\bf Fast Matrix Multiplication.}
\begin{enumerate}[{\bf (a)}]
\item Show that standard matrix multiplication is $O(n^3)$.
\item Show that it is possible to multiply two $2 \times 2$ matrices using only
  7 multiplications. Use the identity
  \[
    \begin{pmatrix}
      a_{11} & a_{12} \\ a_{21} & a_{22}
    \end{pmatrix}
    \begin{pmatrix}
      b_{11} & b_{12} \\ b_{21} & b_{22}
    \end{pmatrix} = C
  \]
  with
  \begin{align*}
    c_{11} &= a_{11}b_{11}+a_{12}b_{21}, \\
    c_{12} &= x+(a_{21}+a_{22})(b_{12}-b_{11}) + (a_{11}+a_{12}-a_{21}-a_{22})b_{22},\\
    c_{21} &= x+(a_{11}-a_{21})(b_{22}-b_{12})-a_{22}(b_{11}-b_{21}-b_{12}+b_{22}),\\
    c_{22} &= x+(a_{11}-a_{21})(b_{22}-b_{12})+(a_{21}+a_{22})(b_{12}-b_{11}).
  \end{align*}
\item Use an inductive argument to show that one may multiply $2^k \times 2^k$
  matrices using only $7^k$ multiplications and fewer than $7^{k+1}$ additions.
  Conclude that two $n \times n$ matrices can be multiplied using $O(n^{\log_2
    7})$ bit operations when all entries of the matrix a less than a constant
  $c$ number of bits.
\end{enumerate}

\end{document}

