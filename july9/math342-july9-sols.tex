\documentclass[a4paper,11pt]{article}

\usepackage[utf8]{inputenc}
\usepackage[english]{babel}
\usepackage{amssymb, amsmath, amsthm, mathrsfs}
\usepackage[left=1.0in,right=1.0in,top=1.0in,bottom=1.0in]{geometry}
\usepackage[T1]{fontenc}
\usepackage{array}
\usepackage{longtable}
\usepackage{multirow}
\usepackage{calc}
\usepackage[inline,shortlabels]{enumitem}
\usepackage{changepage}
\usepackage{booktabs}
\usepackage{capt-of}
\usepackage{subcaption}
\usepackage[leftcaption]{sidecap}
\usepackage[numbers]{natbib}
\usepackage{times}
\usepackage{titlesec}
\usepackage{xcolor}
\usepackage{lineno}
\usepackage{xpatch}
\xpatchcmd\swappedhead{~}{.~}{}{}
\allowdisplaybreaks

\newtheoremstyle{mythm}
{}                % Space above
{}                % Space below
{\itshape}        % Theorem body font % (default is "\upshape")
{1.5em}                % Indent amount
{\scshape}       % Theorem head font % (default is \mdseries)
{.}               % Punctuation after theorem head % default: no punctuation
{0.5em}               % Space after theorem head
{}                % Theorem head spec
\theoremstyle{mythm}


\newtheorem*{theorem*}{Theorem}
\newtheorem{theorem}{Theorem}
\newtheorem{fact}[theorem]{Fact}
\newtheorem{proposition}[theorem]{Proposition}
\newtheorem{lemma}[theorem]{Lemma}
\newtheorem{corollary}[theorem]{Corollary}
\newtheorem{question}[theorem]{Question}
\newtheorem{result}[theorem]{Result}
\newtheorem{observation}[theorem]{Observation}
\newtheorem{conjecture}[theorem]{Conjecture}

\newtheoremstyle{mydef}
{}                % Space above
{}                % Space below
{}        % Theorem body font % (default is "\upshape")
{1.5em}                % Indent amount
{\scshape}       % Theorem head font % (default is \mdseries)
{.}               % Punctuation after theorem head % default: no punctuation
{0.5em}               % Space after theorem head
{}                % Theorem head spec
\theoremstyle{mydef}

\newtheorem{example}[theorem]{Example}
\newtheorem{definition}[theorem]{Definition}
\newtheorem{remark}[theorem]{Remark}
\newtheorem*{remark*}{Remark}

\makeatletter
\renewenvironment{proof}[1][\proofname]{\par
  \pushQED{\qed}%
  \normalfont \topsep6\p@\@plus6\p@\relax
  \trivlist
\item\relax
  {\hspace{1.5em}\itshape
    #1\@addpunct{.}}\hspace\labelsep\ignorespaces
}{%
  \popQED\endtrivlist\@endpefalse
}
\makeatother

\def\Box{\hskip1ex\vbox{\hrule height0.6pt\hbox{%
      \vrule height1.3ex width0.6pt\hskip0.8ex
      \vrule width0.6pt}\hrule height0.6pt
  }}
\renewcommand{\qed}{\Box}

\newcommand{\red}[1]{\textcolor{red}{#1}}
\newcommand{\blue}[1]{\textcolor{blue}{#1}}
\newcommand{\purple}[1]{\textcolor{magenta}{#1}}
\newcommand{\ddet}{\text{det}}
\renewcommand{\pmod}[1]{\text{ (mod $#1$)}}
\newcommand{\mmod}[2]{#1\text{ mod }#2}
\newcommand{\abs}[1]{\left\vert #1 \right\vert}
\newcommand{\C}{\mathbf{C}}
\newcommand{\Z}{\mathbf{Z}}
\newcommand{\N}{\mathbf{N}}
\newcommand{\LL}{\mathscr{G}}
\newcommand{\z}{\mathbin{\ooalign{$\hidewidth i \hidewidth$\cr$\phantom{+}$}}}
\newcommand{\y}{\mathbin{\ooalign{$\hidewidth j \hidewidth$\cr$\phantom{+}$}}}
\newcommand{\gf}{\text{GF}}
\newcommand{\ord}{\text{ord}}

\newcolumntype{R}{>{\scriptsize}r}
\newcolumntype{L}{>{\scriptsize}l}
\newcolumntype{C}{>{\scriptsize}c}

\renewcommand{\citenumfont}[1]{\textbf{#1}}
\renewcommand{\bibnumfmt}[1]{\textbf{#1.}}

\titleformat{\section}{\normalfont\Large\bfseries\centering}{\thesection.}{0.5em}{}
\titleformat{\subsection}{\normalfont\bfseries}{\thesubsection.}{0.5em}{}

\newenvironment{myabstract}{\vspace{1em}\begin{adjustwidth}{3em}{3em}\begin{small}\textbf{Abstract.}}{\end{small}\end{adjustwidth}\vspace{1em}}
\newenvironment{mykeywords}{\vspace{1em}\begin{adjustwidth}{3em}{3em}\begin{small}\textbf{Keywords.}}{\end{small}\end{adjustwidth}\vspace{1em}}

\DeclareCaptionLabelSeparator{custom}{.}
\DeclareCaptionLabelFormat{custom}
{%
  \textsc{#1 #2}
}
\DeclareCaptionFormat{custom}
{%
  #1#2 #3
}
\captionsetup
{
  format=custom,%
  labelformat=custom,%
  labelsep=custom
}

\begin{document}

\begin{center}
  {\Large\bfseries Math 342 Tutorial} \\
  {\normalsize\bf July 9, 2025}
\end{center}

\noindent{\bf Question 1.} Show that if $n = p_1^{2e_1+1} \cdots p_k^{2e_k+1}
q_1^{2f_1} \cdots q_m^{2f_m}$, and if $r$ is an odd prime not dividing $n$, then
\[
  \left( \frac{n}{r} \right) =
  \left( \frac{p_1}{r} \right) \cdots \left( \frac{p_k}{r} \right).
\] \\

\blue{
  Using the fact that the Legendre symbol is totally multiplicative, we see at
  once that
  \[
    \left( \frac{n}{r} \right) = \prod_{i=1}^k\underbrace{\left( \frac{p_1}{r}
    \right)^{2e_1}}_{=1}\left( \frac{p_1}{r}
  \right)\prod_{j=1}^m\underbrace{\left( \frac{q_1}{r} \right)^{2f_j}}_{=1} =
  \prod_{i=1}^k\left( \frac{p_1}{r} \right).
  \]
} \\

\noindent{\bf Question 2.} Show that if $p$ is odd prime that is $3 \pmod{4}$,
then $[(p-1)/2]! \equiv (-1)^t \pmod{p}$ where $t$ is the number of nonquadratic
residues in the range 0 to $\lfloor p/2 \rfloor$ inclusive. \\

\blue{In a similar manner as assignment 4 question 4, we see that $((p-1)/2)!^2
  \equiv -(p-1)! \equiv 1 \pmod{p}$ by an application of Wilson's Theorem. By
  Euler's Criterion, we see that $((p-1)/2)!^{(p-1)/2} \equiv (1 \mid p)(2 \mid
  p) \cdots ((p-1)/2 \mid p) \equiv (-1)^t \pmod{p}$. Since $((p-1)/2)! \equiv
  \pm1 \pmod{p}$, and since $(p-1)/2$ is odd, we're done.} \\

\noindent{\bf Question 3.} Let $p$ be an odd prime. Prove the following
identities.
\begin{enumerate*}[{\bf (a)}]
\item $\sum_{r=1}^{p-1}r(r \mid p) = 0$ if $p \equiv 1 \pmod{4}$.
\item $\sum_{\begin{smallmatrix}r=1 \\ (r \mid p)=1\end{smallmatrix}}^{p-1}r =
  \frac{p(p-1)}{4}$ if $p \equiv 1 \pmod{4}$.
\item $\sum_{r=1}^{p-1}r^2(r \mid p) = p\sum_{r=1}^{p-1}r(r \mid p)$ if $p
  \equiv 3 \pmod{p}$.
\item $\sum_{r=1}^{p-1}r^3(r \mid p) = \frac{3}{2}p\sum_{r=1}^{p-1}r^2(r \mid
  p)$ if $p \equiv 1 \pmod{4}$.
\item $\sum_{r=1}^{p-1}r^4(r \mid p) =$ $2p\sum_{r=1}^{p-1}r^3(r \mid
  p)-p^2\sum_{r=1}^{p-1}r^2(r \mid p)$ if $p \equiv 3\pmod{4}$.
\end{enumerate*}

\blue{
  \begin{enumerate}[{\bf (a)}]
  \item Observe that $(r \mid p) = (-1)^{(p-1)/2}(p-r \mid r)$ (why?). Since $p
    \equiv 1 \pmod{4}$, we then
    have
    \[
      \sum_{r=1}^{p-1}r(r \mid p) = \sum_{r=1}^{p-1}(p-r)(p-r \mid p) =
      \sum_{r=1}^{p-1}(p-r)(r \mid p) = p\sum_{r=1}^{p-1}(r \mid
      p)-\sum_{r=1}^{p-1}r(r \mid p) = -\sum_{r=1}^{p-1}r(r \mid p).
    \]
  \item Since $p \equiv 1 \pmod{4}$, we have
    \[
      \sum_{\begin{smallmatrix}r=1 \\ (r \mid p)=1\end{smallmatrix}}^{p-1}r =
      \sum_{\begin{smallmatrix}r=1 \\ (r \mid p)=1\end{smallmatrix}}^{p-1}(p-r)
      = p\sum_{\begin{smallmatrix}r=1 \\ (r \mid p)=1\end{smallmatrix}}^{p-1}1 -
      \sum_{\begin{smallmatrix}r=1 \\ (r \mid p)=1\end{smallmatrix}}^{p-1}r.
    \]
    Solving for $\sum_{r=1,\,(r \mid p)=1}^{p-1}r$, and using the fact that
    there are $(p-1)/2$ quadratic residues modulo $p$, we obtain the result.
  \item Since $p \equiv 3 \pmod{4}$, we have
    \begin{align*}
      \sum_{r=1}^{p-1}r^2(r \mid p) &= \sum_{r=1}^{p-1}(p-r)^2(p-r \mid p) =
      -\sum_{r=1}^{p-1}(p-r)^2(r \mid p) \\
      &= -p^2\sum_{r=1}^{p-1}(r \mid p) + 2p\sum_{r=1}^{p-1}r(r \mid p) -
        \sum_{r=1}^{p-1}r^2(r \mid p) = 2p\sum_{r=1}^{p-1}r(r \mid p) -
        \sum_{r=1}^{p-1}r^2(r \mid p)
    \end{align*}
    Solving for $\sum_{r=1}^{p-1}r^2(r \mid p)$ gives the result.
  \item For $p \equiv 1 \pmod{4}$, we have that
    \begin{align*}
      \sum_{r=1}^{p-1}r^3(r \mid p) &= \sum_{r=1}^{p-1}(p-r)^3(p-r \mid p)
      = \sum_{r=1}^{p-1}(p-r)^3(r \mid p) \\
      &= p^3\sum_{r=1}^{p-1}(r \mid p) - 3p^2\sum_{r=1}^{p-1}r(r \mid p)
        + 3p\sum_{r=1}^{p-1}r^2(r \mid p) - \sum_{r=1}^{p-1}r^3(r \mid p).
    \end{align*}
    But $\sum_{r=1}^{p-1}(r \mid p) = 0$, and $\sum_{r=1}^{p-1}r(r \mid p) = 0$
    by part (a). Hence, we obtain the result by solving for
    $\sum_{r=1}^{p-1}r^3(r \mid p)$.
  \item Since $p \equiv 3 \pmod{4}$, we have
    \begin{align*}
      \sum_{r=1}^{p-1}r^4(r \mid p) &= \sum_{r=1}^{p-1}(p-r)^4(p-r \mid p) =
      -\sum_{r=1}^{p-1}(p-r)^4(r \mid p) \\
      &= \sum_{i=0}^4(-1)^{i+1}\binom{4}{i}p^{4-j}\sum_{r=1}^{p-1}r^i(r \mid p).
    \end{align*}
    From part (c), we know that $p\sum_{r=1}^{p-1}r(r \mid
    p)=\sum_{r=1}^{p-1}r^2(r \mid p)$. Substituting and solving, we obtain the
    result.
  \end{enumerate}
}

\noindent{\bf Question 4.} Show that if $a$ is a quadratic residue of the prime
$p$, then the solutions of $x^2 \equiv a \pmod{p}$ are
\begin{enumerate*}[{\bf (a)}]
\item $x \equiv \pm a^{n+1} \pmod{p}$ if $p = 4n+3$, or
\item $x \equiv \pm a^{n+1} \text{ or } \pm2^{2n+1}a^{n+1} \pmod{p}$ if $p=8n+5$.
\end{enumerate*}

\blue{
  \begin{enumerate}[{\bf (a)}]
  \item Since $p=4n+3$, we have that
    \[
      x^2 \equiv (\pm a^{n+1})^2 \equiv a^{(p+1)/2} \equiv a^{(p-1)/2}a \equiv a
      \pmod{4}.
    \]
    Hence, the solutions in this case are $\pm a^{n+1} \pmod{p}$.
  \item Because $p \equiv 5 \pmod{8}$, we know that $-1$ is a quadratic residue
    and 2 is a quadratic nonresidues modulo $p$. Next, observe that $(\pm
    a^{n+1})^2 \equiv a^{(p+3)/4}  \pmod{p}$ and $(\pm2^{2n+1}a^{n+1}) \equiv
    2^{(p-1)/2}a^{(p+3)/2} \equiv -a^{(p+3)/2} \pmod{p}$. Since $a$ is a
    quadratic residue, $a^{(p-1)/2} \equiv 1 \pmod{p}$, hence $a^{(p-1)/4}
    \equiv \pm1 \pmod{p}$. But then $\pm a^{(p+3)/4} \equiv a \pmod{p}$.
  \end{enumerate}
}

\noindent{\bf Question 5.} Show there are infinitely many primes of the form
$4k+1$. \\

\blue{
  Suppose there are finitely many such primes, say, $p_1,\,p_2,\dots,\,p_k$.
  Consider $N=4(p_1 \cdots p_k)^2+1$, and let $q$ be a prime divisor of $N$.
  Then $q \neq p_i$ for any $i$, but $N \equiv 0 \pmod{q}$; hence, $4(p_1 \cdots
  p_k)^2 \equiv -1 \pmod{p}$. Therefore, $(-1 \mid q)=1$ which implies $q
  \equiv 1 \pmod{4}$, a contradiction. Therefore, there are infinitely primes
  which are 1 modulo 4.
} \\

\noindent{\bf Question 6.} Show there are infinitely many primes of the
following forms
\begin{enumerate*}[{\bf (a)}]
\item $8k+3$,
\item $8k+5$, and
\item $8k+7$.
\end{enumerate*}
[Hint: For each part, assume there are only finitely many primes
$p_1,\,p_2,\dots,\,p_k$ of the required form. For (a), consider $(p_1 \cdots
p_k)^2+2$; for (b), consider $(p_1 \cdots p_k)^2+4$; for (c), consider $(4p_1
\cdots p_k)^2-2$. Use what you know about $(-1 \mid p)$ and $(2 \mid p)$.]

\blue{
  \begin{enumerate}[{\bf (a)}]
  \item Let $N=(p_1 \cdots p_k)^2+2$; then $N \equiv 3 \pmod{8}$. Note that the
    product of two integers which are 1 mod 8 is again 1 mod 8. Therefore, $N$
    has an odd prime divisor $q \not\equiv 1 \pmod{8}$. Since $N \equiv 0
    \pmod{q}$, we have that $(p_1 \cdots p_k)^2 \equiv -2 \pmod{q}$, hence $(-2
    \mid q) = 1$. Therefore, $q \equiv 1 \text{ or }3 \pmod{8}$. But we have
    excluded the case $q \equiv 1 \pmod{8}$. But we easily see that $q \neq p_i$
    for all $i$. We have, therefore, reached a contradiction.
  \item Let $N=(p_1 \cdots p_k)^2+4$; then $N \equiv 5 \pmod{8}$. As before,
    there is an odd prime divisor $q \not\equiv 1 \pmod{8}$ and $q \neq p_i$ for
    any $i$. But then $(p_1 \cdots p_k)^2 \equiv -4 \pmod{q}$. Since $4$ is a
    quadratic residue, so $-1$ must also be a quadratic residue. Hence $q \equiv
    1 \pmod{4}$. But $q \not\equiv 1 \pmod{8}$, hence $q \equiv 5 \pmod{8}$. We
    have reached our contradiction.
  \item Let $N=(4p_1 \cdots p_k)^2-2$. Then $N/2 \equiv 7 \pmod{8}$. and must
    have an odd prime divisor $q \not\equiv 1 \pmod{8}$. We have that $2 \equiv
    (4p_1 \cdots p_k)^2 \pmod{q}$, so $(2 \mid q)=1$ and $q \equiv \pm
    1\pmod{8}$. Hence, $q \equiv -1 \pmod{8}$, and we again reach a contradiction.
  \end{enumerate}
}

\noindent{\bf Question 7.} Show the following.
\begin{enumerate*}[{\bf (a)}]
\item If $p=4k+1$ is a prime, then there is an integer $x$ such that $mp=1+x^2$
  where $0 < m < p$.
\item If $p$ is an odd prime, then there are integers $x$ and $y$ such that
  $1+x^2+y^2 = mp$ where again $0 < m < p$.
\end{enumerate*}

\blue{
  \begin{enumerate}[{\bf (a)}]
  \item Since $p \equiv 1 \pmod{4}$, we know that $-1$ is a quadratic residue of
    $p$ and hence is one of $1^2,\,2^2,\dots,\,[(p-1)/2]^2$, say, $x^2$.
    But, $0 < 1+x^2 < 1+(p/2)^2 < p^2$.
  \item The $(p+1)/2$ numbers $x : 0 \leqq x \leqq (p-1)/2$ are incongruent.
    Also, the $(p+1)/2$ numbers $-1-y^2 : 0 \leqq y \leqq (p-1)/2$ are
    incongruent. The cardinalities of these two sets sum to $p+1$; as there are
    only $p$ residues modulo $p$, one must reside in both sets. Additionally, $0
    < 1+x^2+y^2 < 1+2(p/2)^2 < p^2$.
  \end{enumerate}
}

\noindent{\bf Question 8.} Determine those primes for which 7 is a quadratic
residue. \\

\blue{By quadratic reciprocity, we have that $(7 \mid p) = (-1)^{(p-1)/2}(p \mid
  7)$. Suppose first that $p \equiv 1 \pmod{4}$: if $p \equiv 1 \pmod{7}$, then
  $p \equiv 1 \pmod{28}$; if $p \equiv 2 \pmod{7}$, then $p \equiv 9 \pmod{28}$;
if $p \equiv 4 \pmod{7}$, then $p \equiv -3 \pmod{28}$. Suppose next that $p
\equiv 3 \pmod{4}$: if $p \equiv 3 \pmod{7}$, then $p \equiv 3 \pmod{28}$; if $p
\equiv 5 \pmod{7}$, then $p \equiv -9 \pmod{28}$; if $p \equiv 6 \pmod{7}$, then
$p \equiv -1 \pmod{28}$. We have, therefore, shown the following
\[
  \left( \frac{7}{p} \right) =
  \begin{cases}
    1 & \text{if $p \equiv \pm1,\,\pm3,\,\pm9 \pmod{28}$,} \\
    -1 & \text{if $p \equiv \pm5,\,\pm7,\,\pm11,\,\pm13 \pmod{28}$.}
  \end{cases}
\]
}

Let $a$ and $n$ be positive integers with $n$ odd, and let $n=p_1^{e_1} \cdots
p_k^{e_k}$ be the prime power factorization of $n$. The Jocabi symbol is defined
as $(a \mid n) = (a \mid p_1)^{e_1} \cdots (a \mid p_k)^{e_k}$. \\

\noindent{\bf Question 9.} Prove the following properties of the Jacobi symbol.
\begin{enumerate*}[{\bf (a)}]
\item If $a \equiv b \pmod{n}$, then $(a \mid n) = (b \mid n)$;
\item $(ab \mid n) = (a \mid n)(b \mid n)$;
\item $(-1 \mid n) = (-1)^{(n-1)/2}$;
\item $(2 \mid n) = (-1)^{(n^2-1)/8}$.
\end{enumerate*} [Hint: Use the fact that $(1+(x-1))(1+(y-1))=xy$.]

\blue{
  \begin{enumerate}[{\bf (a)}]
  \item We know that if $a \equiv b \pmod{p}$, then $(a \mid p) = (b \mid p)$.
    Hence,
    \[
      \left( \frac{a}{n} \right) = \prod_i \left( \frac{a}{p_i} \right)^{e_i} =
      \prod_i \left( \frac{b}{p_i} \right)^{e_i} = \left( \frac{b}{n} \right).
    \]
  \item By the totally multiplicative property of the Legendre symbol, we have
    \[
      \left( \frac{ab}{n} \right) = \prod_i \left( \frac{ab}{p_i} \right)^{e_i}
      = \prod_i\left( \frac{a}{p_i} \right)^{e_i}\left( \frac{b}{p_i}
      \right)^{e_i} = \left( \prod_i\left( \frac{a}{p_i} \right)^{e_i} \right)
      \left( \prod_i\left( \frac{b}{p_i} \right)^{e_i} \right)
      = \left( \frac{a}{n} \right)\left( \frac{b}{n} \right).
    \]
  \item We have
    \[
      \left( \frac{-1}{n} \right) = \prod_i\left( \frac{-1}{p_i} \right)^{e_i} =
      (-1)^{\frac{1}{2}\sum_i(p_i-1)e_i}.
    \]
    An obvious induction using the hint implies that
    $n=\prod_i(1+(p_i-1))^{e_i}$. Furthermore, $(1+(p_i-1))^{e_i} \equiv
    1+e_i(p_i-1) \pmod{4}$ and $(1+(p_i-1))^{e_i}(1+(p_j-1))^{e_j} \equiv 1 +
    e_i(p_i-1) + e_j(p_j-1) \pmod{4}$. Another clear induction shows
    $n=\prod_i(1+(p_i-1))^{e_i} \equiv 1+\sum_ie_i(p_i-1) \pmod{4}$. Since then
    $(n-1)/2 \equiv \sum_ie_i(p_i-1)/2 \pmod{2}$, the result now follows.
  \item Similar reasoning as that used in part (c) yields
    \[
      \left( \frac{2}{n} \right) = (-1)^{\frac{1}{8}\sum_ie_i(p_i^2-1)}.
    \]
    As before, $n^2 \equiv 1+\sum_ie_i(p_i^2-1) \pmod{4}$. The result follows as
    above.
  \end{enumerate}
}

\noindent{\bf Question 10.} Prove quadratic reciprocity holds for the Jacobi
symbol, i.e., $(n \mid m)(m \mid n) = (-1)^{(m-1)(n-1)/4}$. \\

\blue{Let $n=\prod_{i=1}^sp_i^{e_i}$ and $m=\prod_{i=1}^tq_i^{f_i}$. By the
  definition of the Jacobi symbol and the total multiplicativity of the Legendre
  symbol,
  \[
    \left( \frac{m}{n} \right)\left( \frac{n}{m} \right) =
    \prod_{i=1}^s\prod_{j=1}^t \left[ \left( \frac{p_i}{q_j} \right)\left(
        \frac{q_j}{p_i} \right) \right]^{e_if_j}.
  \]
  Quadratic reciprocity now gives
  \[
    \left( \frac{m}{n} \right)\left( \frac{n}{m} \right) =
    \prod_{i=1}^s\prod_{j=1}^t(-1)^{(p_i-1)(q_j-1)e_if_j/4} =
    (-1)^{\sum_{i=1}^s\sum_{j=1}^te_i\frac{p_i-1}{2}f_j\frac{q_j-1}{2}} =
    (-1)^{(\sum_{i=1}^se_i\frac{p_i-1}{2})(\sum_{j=1}^tf_j\frac{q_j-1}{2})}.
  \]
  As in question 9, we have that
  \begin{align*}
    \sum_{i=1}^se_i\frac{p_i-1}{2} &\equiv \frac{n-1}{2} \pmod{2}, \\
    \sum_{j=1}^tf_j\frac{q_j-1}{2} &\equiv \frac{m-1}{2} \pmod{2}.
  \end{align*}
  Therefore,
  \[
    \left( \frac{m}{n} \right)\left( \frac{n}{m} \right) = (-1)^{(n-1)(m-1)/4}.
  \]
}

\end{document}

