\documentclass[a4paper,11pt]{article}

\usepackage[utf8]{inputenc}
\usepackage[english]{babel}
\usepackage{amssymb, amsmath, amsthm, mathrsfs}
\usepackage[left=1.0in,right=1.0in,top=1.0in,bottom=1.0in]{geometry}
\usepackage[T1]{fontenc}
\usepackage{array}
\usepackage{longtable}
\usepackage{multirow}
\usepackage{calc}
\usepackage[inline,shortlabels]{enumitem}
\usepackage{changepage}
\usepackage{booktabs}
\usepackage{capt-of}
\usepackage{subcaption}
\usepackage[leftcaption]{sidecap}
\usepackage[numbers]{natbib}
\usepackage{times}
\usepackage{titlesec}
\usepackage{xcolor}
\usepackage{lineno}
\usepackage{xpatch}
\xpatchcmd\swappedhead{~}{.~}{}{}
\allowdisplaybreaks

\newtheoremstyle{mythm}
{}                % Space above
{}                % Space below
{\itshape}        % Theorem body font % (default is "\upshape")
{1.5em}                % Indent amount
{\scshape}       % Theorem head font % (default is \mdseries)
{.}               % Punctuation after theorem head % default: no punctuation
{0.5em}               % Space after theorem head
{}                % Theorem head spec
\theoremstyle{mythm}


\newtheorem*{theorem*}{Theorem}
\newtheorem{theorem}{Theorem}
\newtheorem{fact}[theorem]{Fact}
\newtheorem{proposition}[theorem]{Proposition}
\newtheorem{lemma}[theorem]{Lemma}
\newtheorem{corollary}[theorem]{Corollary}
\newtheorem{question}[theorem]{Question}
\newtheorem{result}[theorem]{Result}
\newtheorem{observation}[theorem]{Observation}
\newtheorem{conjecture}[theorem]{Conjecture}

\newtheoremstyle{mydef}
{}                % Space above
{}                % Space below
{}        % Theorem body font % (default is "\upshape")
{1.5em}                % Indent amount
{\scshape}       % Theorem head font % (default is \mdseries)
{.}               % Punctuation after theorem head % default: no punctuation
{0.5em}               % Space after theorem head
{}                % Theorem head spec
\theoremstyle{mydef}

\newtheorem{example}[theorem]{Example}
\newtheorem{definition}[theorem]{Definition}
\newtheorem{remark}[theorem]{Remark}
\newtheorem*{remark*}{Remark}

\makeatletter
\renewenvironment{proof}[1][\proofname]{\par
  \pushQED{\qed}%
  \normalfont \topsep6\p@\@plus6\p@\relax
  \trivlist
\item\relax
  {\hspace{1.5em}\itshape
    #1\@addpunct{.}}\hspace\labelsep\ignorespaces
}{%
  \popQED\endtrivlist\@endpefalse
}
\makeatother

\def\Box{\hskip1ex\vbox{\hrule height0.6pt\hbox{%
      \vrule height1.3ex width0.6pt\hskip0.8ex
      \vrule width0.6pt}\hrule height0.6pt
  }}
\renewcommand{\qed}{\Box}

\newcommand{\red}[1]{\textcolor{red}{#1}}
\newcommand{\blue}[1]{\textcolor{blue}{#1}}
\newcommand{\purple}[1]{\textcolor{magenta}{#1}}
\newcommand{\ddet}{\text{det}}
\renewcommand{\pmod}[1]{\text{ (mod $#1$)}}
\newcommand{\mmod}[2]{#1\text{ mod }#2}
\newcommand{\abs}[1]{\left\vert #1 \right\vert}
\newcommand{\C}{\mathbf{C}}
\newcommand{\Z}{\mathbf{Z}}
\newcommand{\N}{\mathbf{N}}
\newcommand{\LL}{\mathscr{G}}
\newcommand{\z}{\mathbin{\ooalign{$\hidewidth i \hidewidth$\cr$\phantom{+}$}}}
\newcommand{\y}{\mathbin{\ooalign{$\hidewidth j \hidewidth$\cr$\phantom{+}$}}}
\newcommand{\gf}{\text{GF}}
\newcommand{\ord}{\text{ord}}

\newcolumntype{R}{>{\scriptsize}r}
\newcolumntype{L}{>{\scriptsize}l}
\newcolumntype{C}{>{\scriptsize}c}

\renewcommand{\citenumfont}[1]{\textbf{#1}}
\renewcommand{\bibnumfmt}[1]{\textbf{#1.}}

\titleformat{\section}{\normalfont\Large\bfseries\centering}{\thesection.}{0.5em}{}
\titleformat{\subsection}{\normalfont\bfseries}{\thesubsection.}{0.5em}{}

\newenvironment{myabstract}{\vspace{1em}\begin{adjustwidth}{3em}{3em}\begin{small}\textbf{Abstract.}}{\end{small}\end{adjustwidth}\vspace{1em}}
\newenvironment{mykeywords}{\vspace{1em}\begin{adjustwidth}{3em}{3em}\begin{small}\textbf{Keywords.}}{\end{small}\end{adjustwidth}\vspace{1em}}

\DeclareCaptionLabelSeparator{custom}{.}
\DeclareCaptionLabelFormat{custom}
{%
  \textsc{#1 #2}
}
\DeclareCaptionFormat{custom}
{%
  #1#2 #3
}
\captionsetup
{
  format=custom,%
  labelformat=custom,%
  labelsep=custom
}

\begin{document}

\begin{center}
  {\Large\bfseries Math 342 Tutorial} \\
  {\normalsize\bf July 9, 2025}
\end{center}

\noindent{\bf Question 1.} Show that if $n = p_1^{2e_1+1} \cdots p_k^{2e_k+1}
q_1^{2f_1} \cdots q_m^{2f_m}$, and if $r$ is an odd prime not dividing $n$, then
\[
  \left( \frac{n}{r} \right) =
  \left( \frac{p_1}{r} \right) \cdots \left( \frac{p_k}{r} \right).
\] \\

\noindent{\bf Question 2.} Show that if $p$ is odd prime that is $3 \pmod{4}$,
then $[(p-1)/2]! \equiv (-1)^t \pmod{p}$ where $t$ is the number of nonquadratic
residues in the range 0 to $\lfloor p/2 \rfloor$ inclusive. \\

\noindent{\bf Question 3.} Let $p$ be an odd prime. Prove the following
identities.
\begin{enumerate*}[{\bf (a)}]
\item $\sum_{r=1}^{p-1}r(r \mid p) = 0$ if $p \equiv 1 \pmod{4}$.
\item $\sum_{\begin{smallmatrix}r=1 \\ (r \mid p)=1\end{smallmatrix}}^{p-1}r =
  \frac{p(p-1)}{4}$ if $p \equiv 1 \pmod{4}$.
\item $\sum_{r=1}^{p-1}r^2(r \mid p) = p\sum_{r=1}^{p-1}r(r \mid p)$ if $p
  \equiv 3 \pmod{p}$.
\item $\sum_{r=1}^{p-1}r^3(r \mid p) = \frac{3}{2}p\sum_{r=1}^{p-1}r^2(r \mid
  p)$ if $p \equiv 1 \pmod{4}$.
\item $\sum_{r=1}^{p-1}r^4(r \mid p) =$ $2p\sum_{r=1}^{p-1}r^3(r \mid
  p)-p^2\sum_{r=1}^{p-1}r^2(r \mid p)$ if $p \equiv 1\pmod{4}$.
\end{enumerate*} \\

\noindent{\bf Question 4.} Show that if $a$ is a quadratic residue of the prime
$p$, then the solutions of $x^2 \equiv a \pmod{p}$ are
\begin{enumerate*}[{\bf (a)}]
\item $x \equiv \pm a^{n+1} \pmod{p}$ if $p = 4n+3$, or
\item $x \equiv \pm a^{n+1} \text{ or } \pm2^{2n+1}a^{n+1} \pmod{p}$ if $p=8n+5$.
\end{enumerate*} \\

\noindent{\bf Question 5.} Show there are infinitely many primes of the form
$4k+1$. \\

\noindent{\bf Question 6.} Show there are infinitely many primes of the
following forms
\begin{enumerate*}[{\bf (a)}]
\item $8k+3$,
\item $8k+5$, and
\item $8k+7$.
\end{enumerate*}
[Hint: For each part, assume there are only finitely many primes
$p_1,\,p_2,\dots,\,p_k$ of the required form. For (a), consider $(p_1 \cdots
p_k)^2+2$; for (b), consider $(p_1 \cdots p_k)^2+4$; for (c), consider $(4p_1
\cdots p_k)^2-2$. Use what you know about $(-1 \mid p)$ and $(2 \mid p)$.] \\

\noindent{\bf Question 7.} Show the following.
\begin{enumerate*}[{\bf (a)}]
\item If $p=4k+1$ is a prime, then there is an integer $x$ such that $mp=1+x^2$
  where $0 < m < p$.
\item If $p$ is an odd prime, then there are integers $x$ and $y$ such that
  $1+x^2+y^2 = mp$ where again $0 < m < p$.
\end{enumerate*} \\

\noindent{\bf Question 8.} Determine those primes for which 7 is a quadratic
residue. \\

Let $a$ and $n$ be positive integers with $n$ odd, and let $n=p_1^{e_1} \cdots
p_k^{e_k}$ be the prime power factorization of $n$. The Jocabi symbol is defined
as $(a \mid n) = (a \mid p_1)^{e_1} \cdots (a \mid p_k)^{e_k}$. \\

\noindent{\bf Question 9.} Prove the following properties of the Jacobi symbol.
\begin{enumerate*}[{\bf (a)}]
\item If $a \equiv b \pmod{n}$, then $(a \mid n) = (b \mid n)$;
\item $(ab \mid n) = (a \mid n)(b \mid n)$;
\item $(-1 \mid n) = (-1)^{(n-1)/2}$;
\item $(2 \mid n) = (-1)^{(n^2-1)/8}$.
\end{enumerate*} [Hint: Use the fact that $(1+(x-1))(1+(y-1))=xy$.] \\

\noindent{\bf Question 10.} Prove quadratic reciprocity holds for the Jacobi
symbol, i.e., $(n \mid m)(m \mid n) = (-1)^{(m-1)(n-1)/4}$.

\end{document}

