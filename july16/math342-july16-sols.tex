\documentclass[a4paper,11pt]{article}

\usepackage[utf8]{inputenc}
\usepackage[english]{babel}
\usepackage{amssymb, amsmath, amsthm, mathrsfs}
\usepackage[left=1.0in,right=1.0in,top=1.0in,bottom=1.0in]{geometry}
\usepackage[T1]{fontenc}
\usepackage{array}
\usepackage{longtable}
\usepackage{multirow}
\usepackage{calc}
\usepackage[inline,shortlabels]{enumitem}
\usepackage{changepage}
\usepackage{booktabs}
\usepackage{capt-of}
\usepackage{subcaption}
\usepackage[leftcaption]{sidecap}
\usepackage[numbers]{natbib}
\usepackage{times}
\usepackage{titlesec}
\usepackage{xcolor}
\usepackage{lineno}
\usepackage{xpatch}
\xpatchcmd\swappedhead{~}{.~}{}{}
\allowdisplaybreaks

\newtheoremstyle{mythm}
{}                % Space above
{}                % Space below
{\itshape}        % Theorem body font % (default is "\upshape")
{1.5em}                % Indent amount
{\scshape}       % Theorem head font % (default is \mdseries)
{.}               % Punctuation after theorem head % default: no punctuation
{0.5em}               % Space after theorem head
{}                % Theorem head spec
\theoremstyle{mythm}


\newtheorem*{theorem*}{Theorem}
\newtheorem{theorem}{Theorem}
\newtheorem{fact}[theorem]{Fact}
\newtheorem{proposition}[theorem]{Proposition}
\newtheorem{lemma}[theorem]{Lemma}
\newtheorem{corollary}[theorem]{Corollary}
\newtheorem{question}[theorem]{Question}
\newtheorem{result}[theorem]{Result}
\newtheorem{observation}[theorem]{Observation}
\newtheorem{conjecture}[theorem]{Conjecture}

\newtheoremstyle{mydef}
{}                % Space above
{}                % Space below
{}        % Theorem body font % (default is "\upshape")
{1.5em}                % Indent amount
{\scshape}       % Theorem head font % (default is \mdseries)
{.}               % Punctuation after theorem head % default: no punctuation
{0.5em}               % Space after theorem head
{}                % Theorem head spec
\theoremstyle{mydef}

\newtheorem{example}[theorem]{Example}
\newtheorem{definition}[theorem]{Definition}
\newtheorem{remark}[theorem]{Remark}
\newtheorem*{remark*}{Remark}

\makeatletter
\renewenvironment{proof}[1][\proofname]{\par
  \pushQED{\qed}%
  \normalfont \topsep6\p@\@plus6\p@\relax
  \trivlist
\item\relax
  {\hspace{1.5em}\itshape
    #1\@addpunct{.}}\hspace\labelsep\ignorespaces
}{%
  \popQED\endtrivlist\@endpefalse
}
\makeatother

\def\Box{\hskip1ex\vbox{\hrule height0.6pt\hbox{%
      \vrule height1.3ex width0.6pt\hskip0.8ex
      \vrule width0.6pt}\hrule height0.6pt
  }}
\renewcommand{\qed}{\Box}

\newcommand{\red}[1]{\textcolor{red}{#1}}
\newcommand{\blue}[1]{\textcolor{blue}{#1}}
\newcommand{\purple}[1]{\textcolor{magenta}{#1}}
\newcommand{\ddet}{\text{det}}
\renewcommand{\pmod}[1]{\text{ (mod $#1$)}}
\newcommand{\mmod}[2]{#1\text{ mod }#2}
\newcommand{\abs}[1]{\left\vert #1 \right\vert}
\newcommand{\C}{\mathbf{C}}
\newcommand{\Z}{\mathbf{Z}}
\newcommand{\N}{\mathbf{N}}
\newcommand{\LL}{\mathscr{G}}
\newcommand{\z}{\mathbin{\ooalign{$\hidewidth i \hidewidth$\cr$\phantom{+}$}}}
\newcommand{\y}{\mathbin{\ooalign{$\hidewidth j \hidewidth$\cr$\phantom{+}$}}}
\newcommand{\gf}{\text{GF}}
\newcommand{\ord}{\text{ord}}

\newcolumntype{R}{>{\scriptsize}r}
\newcolumntype{L}{>{\scriptsize}l}
\newcolumntype{C}{>{\scriptsize}c}

\renewcommand{\citenumfont}[1]{\textbf{#1}}
\renewcommand{\bibnumfmt}[1]{\textbf{#1.}}

\titleformat{\section}{\normalfont\Large\bfseries\centering}{\thesection.}{0.5em}{}
\titleformat{\subsection}{\normalfont\bfseries}{\thesubsection.}{0.5em}{}

\newenvironment{myabstract}{\vspace{1em}\begin{adjustwidth}{3em}{3em}\begin{small}\textbf{Abstract.}}{\end{small}\end{adjustwidth}\vspace{1em}}
\newenvironment{mykeywords}{\vspace{1em}\begin{adjustwidth}{3em}{3em}\begin{small}\textbf{Keywords.}}{\end{small}\end{adjustwidth}\vspace{1em}}

\DeclareCaptionLabelSeparator{custom}{.}
\DeclareCaptionLabelFormat{custom}
{%
  \textsc{#1 #2}
}
\DeclareCaptionFormat{custom}
{%
  #1#2 #3
}
\captionsetup
{
  format=custom,%
  labelformat=custom,%
  labelsep=custom
}

\begin{document}

\begin{center}
  {\Large\bfseries Math 342 Tutorial} \\
  {\normalsize\bf July 16, 2025}
\end{center}

\noindent{\bf Question 1.} Show the following:
\begin{enumerate*}[{\bf (a)}]
\item If $n$ is an Euler pseudo prime to the bases $a$ and $b$, then $n$ is an
  Euler pseudoprime to the base $ab$.
\item If $n$ is an Euler pseudoprime to the base $b$, then $n$ is also an Euler
  pseudoprime to the base $n-b$.
\item If $n \equiv 5 \pmod{8}$ and $n$ is an Euler pseudoprime to the base 2,
  then $n$ is a strong pseudoprime to the base 2.
\item If $n \equiv 5 \pmod{12}$ and $n$ is an Euler pseudoprime to the base 3,
  then $n$ is a strong pseudoprime to the base 3.
\end{enumerate*}

\blue{
  \begin{enumerate}[{\bf (a)}]
  \item We have $(ab)^{(n-1)/2} \equiv a^{(n-1)/2}b^{(n-1)/2} \equiv (a \mid
    n)(b \mid n) \equiv (ab \mid n) \pmod{n}$. This shows that $n$ is an Euler
    pseudoprime to the base $ab$.
  \item Observe $(n-b)^{(n-1)/2} \equiv (-1)^{(n-1)/2}b^{(n-1)/2} \equiv (-1
    \mid n)(b \mid n) \equiv (-b \mid n) \equiv (n-b \mid n) \pmod{n}$,
    whereupon $n$ is an Euler pseudoprime to the base $n-b$.
  \item By assumption $2^{(n-1)/2} \equiv (2 \mid n) \equiv -1 \pmod{n}$. Also
    by assumption, we have $n-1 = 2^2t$ with $t$ odd. Thus, $-1 \equiv
    2^{(n-1)/2} \equiv 2^{2t} \pmod{n}$. But this means that $n$ is a strong
    pseudoprime to the base $2$.
  \item By assumption, $3^{(n-1)/2} \equiv (3 \mid n) \equiv -1 \pmod{n}$. Also
    by assumption, $n-1 = 12k+4 = 2^2(3k+1)$. Then $-1 \equiv 3^{(n-1)/2} \equiv
    3^{2(3k+1)} \pmod{n}$, and $n$ passes Miller's test.
  \end{enumerate}
}

\noindent{\bf Question 2.} Show that if $n=p_1 \cdots p_k$ is square-free, and
if each $(p_i-1) \mid (n-1)$, then $n$ is a Carmichael number. \\

\blue{Let $b$ be a positive integer with $(b,\,n)=1$, Then $b^{p_i-1} \equiv 1
  \pmod{p_i}$ for each $p_i$. By assumption, there are integers $t_i$ such that
  $t_i(p_i-1) = n-1$; but then $b^{n-1} \equiv 1 \pmod{p_i}$ for each $p_i$. It
  follows, therefore, that $b^{n-1} \equiv 1 \pmod{n}$.} \\

\noindent{\bf Question 3.} Show the following:
\begin{enumerate*}[{\bf (a)}]
\item Show that if $n$ is a pseudoprime to the bases $a$ and $b$, then $n$ is a
  pseudoprime to the base $ab$.
\item Suppose that $(n,\,a)=1$. If $n$ is a pseudoprime to the base $a$, then
  $n$ is a pseudoprime to the base $\overline{a}$ where $\overline{a}$ is the
  inverse of $a$ modulo $n$.
\end{enumerate*}

\blue{
  \begin{enumerate}[{\bf (a)}]
  \item Note $(ab)^{n-1} \equiv a^{n-1}b^{n-1} \equiv 1 \cdot 1 \equiv 1
    \pmod{n}$, hence $n$ is a pseudoprime to the base $ab$.
  \item Since $\overline{a}$ is the modular inverse of $a$, we have that
    $\overline{a^{n-1}} = \overline{a}^{n-1}$ is the modular inverse of
    $a^{n-1}$. Then $a^{n-1} \equiv 1 \pmod{n}$ if and only if $1 \equiv
    \overline{a}^{n-1} \pmod{n}$.
  \end{enumerate}
}

\noindent{\bf Question 4.} Show that if $n=(a^{2p}-1)/(a^2-1)$, where $a>1$ is
an integer, and $p$ an odd prime not dividing $a(a^2-1)$, then $n$ is a
pseudoprime to the base $a$. Conclude there are infinitely many pseudoprimes to
any any base. [Hint: To establish that $a^{n-1} \equiv 1 \pmod{n}$, show that
$2p \mid n-1$, and demonstrate that $a^{2p} \equiv 1 \pmod{n}$.] \\

\blue{Note that $n-1 = a^2(a^{2(p-1)}-1)/(a^2-1)$. Since $a^{2(p-1)} \equiv 1
  \pmod{p}$, we have that $n-1 \equiv 0 \pmod{p}$. Next, we write
  $a^2(a^{2(p-1)}-1)/(a^2-1) = a^2(1+a^2+\cdots+a^{2(p-2)})$. Hence, if $a$ is
  odd then $n-1 \equiv 0 \pmod{2}$. In all cases, then, we have that $n-1 \equiv
  0 \pmod{2p}$. Now, $a^{2p}-1 \equiv n(a^2-1) \equiv 0 \pmod{n}$, whereupon
  $a^{n-1} \equiv a^{2pk} \equiv 1^k \equiv 1 \pmod{n}$ for some integer $k$.} \\

\noindent{\bf Question 5.} Show that if $n$ is a Carmichael number, then $n$ is
square free. \\

\blue{Let $n$ be a Carmichael number. Suppose there is a prime $p$ such that $n
  = p^tm$ with $(p,\,m)=1$ and $t > 1$. Let $b=x$ be a solution to the system
of linear congruences $x \equiv p^{t-1}+1 \pmod{p^t}$ and $x \equiv 1 \pmod{m}$.
Then, since $(b,\,p) = 1 = (b,\,m)$, we have that $(b,\,n) = 1$. If it were the
case that $b \equiv 1 \pmod{n}$, then $b \equiv 1 \pmod{p^t}$, a contradiction.
Thus, $b \not\equiv 1 \pmod{n}$. Next note that $b^n \equiv (p^{t-1}+1)^n \equiv
1 \pmod{p^t}$ by the binomial theorem and the fact that $p^t \mid n$. Since also
$b \equiv 1 \pmod{m}$, we have that $b^n \equiv 1 \pmod{n}$. But then $b^n
\not\equiv b \pmod{n}$ whereupon $n$ is not a Carmichael number. Since this
contradicts our original assumption, it follows that $n$ must be square-free
whenever $n$ is a Carmichael number.}

\end{document}

