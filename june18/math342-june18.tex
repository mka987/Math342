\documentclass[a4paper,11pt]{article}

\usepackage[utf8]{inputenc}
\usepackage[english]{babel}
\usepackage{amssymb, amsmath, amsthm, mathrsfs}
\usepackage[left=1.0in,right=1.0in,top=1.0in,bottom=1.0in]{geometry}
\usepackage[T1]{fontenc}
\usepackage{array}
\usepackage{longtable}
\usepackage{multirow}
\usepackage{calc}
\usepackage[inline,shortlabels]{enumitem}
\usepackage{changepage}
\usepackage{booktabs}
\usepackage{capt-of}
\usepackage{subcaption}
\usepackage[leftcaption]{sidecap}
\usepackage[numbers]{natbib}
\usepackage{times}
\usepackage{titlesec}
\usepackage{xcolor}
\usepackage{lineno}
\usepackage{xpatch}
\xpatchcmd\swappedhead{~}{.~}{}{}
\allowdisplaybreaks

\newtheoremstyle{mythm}
{}                % Space above
{}                % Space below
{\itshape}        % Theorem body font % (default is "\upshape")
{1.5em}                % Indent amount
{\scshape}       % Theorem head font % (default is \mdseries)
{.}               % Punctuation after theorem head % default: no punctuation
{0.5em}               % Space after theorem head
{}                % Theorem head spec
\theoremstyle{mythm}


\newtheorem*{theorem*}{Theorem}
\newtheorem{theorem}{Theorem}
\newtheorem{fact}[theorem]{Fact}
\newtheorem{proposition}[theorem]{Proposition}
\newtheorem{lemma}[theorem]{Lemma}
\newtheorem{corollary}[theorem]{Corollary}
\newtheorem{question}[theorem]{Question}
\newtheorem{result}[theorem]{Result}
\newtheorem{observation}[theorem]{Observation}
\newtheorem{conjecture}[theorem]{Conjecture}

\newtheoremstyle{mydef}
{}                % Space above
{}                % Space below
{}        % Theorem body font % (default is "\upshape")
{1.5em}                % Indent amount
{\scshape}       % Theorem head font % (default is \mdseries)
{.}               % Punctuation after theorem head % default: no punctuation
{0.5em}               % Space after theorem head
{}                % Theorem head spec
\theoremstyle{mydef}

\newtheorem{example}[theorem]{Example}
\newtheorem{definition}[theorem]{Definition}
\newtheorem{remark}[theorem]{Remark}
\newtheorem*{remark*}{Remark}

\makeatletter
\renewenvironment{proof}[1][\proofname]{\par
  \pushQED{\qed}%
  \normalfont \topsep6\p@\@plus6\p@\relax
  \trivlist
\item\relax
  {\hspace{1.5em}\itshape
    #1\@addpunct{.}}\hspace\labelsep\ignorespaces
}{%
  \popQED\endtrivlist\@endpefalse
}
\makeatother

\def\Box{\hskip1ex\vbox{\hrule height0.6pt\hbox{%
      \vrule height1.3ex width0.6pt\hskip0.8ex
      \vrule width0.6pt}\hrule height0.6pt
  }}
\renewcommand{\qed}{\Box}

\newcommand{\red}[1]{\textcolor{red}{#1}}
\newcommand{\blue}[1]{\textcolor{blue}{#1}}
\newcommand{\purple}[1]{\textcolor{magenta}{#1}}
\newcommand{\ddet}{\text{det}}
\renewcommand{\pmod}[1]{\text{ (mod $#1$)}}
\newcommand{\mmod}[2]{#1\text{ mod }#2}
\newcommand{\abs}[1]{\left\vert #1 \right\vert}
\newcommand{\C}{\mathbf{C}}
\newcommand{\Z}{\mathbf{Z}}
\newcommand{\N}{\mathbf{N}}
\newcommand{\LL}{\mathscr{G}}
\newcommand{\z}{\mathbin{\ooalign{$\hidewidth i \hidewidth$\cr$\phantom{+}$}}}
\newcommand{\y}{\mathbin{\ooalign{$\hidewidth j \hidewidth$\cr$\phantom{+}$}}}
\newcommand{\gf}{\text{GF}}

\newcolumntype{R}{>{\scriptsize}r}
\newcolumntype{L}{>{\scriptsize}l}
\newcolumntype{C}{>{\scriptsize}c}

\renewcommand{\citenumfont}[1]{\textbf{#1}}
\renewcommand{\bibnumfmt}[1]{\textbf{#1.}}

\titleformat{\section}{\normalfont\Large\bfseries\centering}{\thesection.}{0.5em}{}
\titleformat{\subsection}{\normalfont\bfseries}{\thesubsection.}{0.5em}{}

\newenvironment{myabstract}{\vspace{1em}\begin{adjustwidth}{3em}{3em}\begin{small}\textbf{Abstract.}}{\end{small}\end{adjustwidth}\vspace{1em}}
\newenvironment{mykeywords}{\vspace{1em}\begin{adjustwidth}{3em}{3em}\begin{small}\textbf{Keywords.}}{\end{small}\end{adjustwidth}\vspace{1em}}

\DeclareCaptionLabelSeparator{custom}{.}
\DeclareCaptionLabelFormat{custom}
{%
  \textsc{#1 #2}
}
\DeclareCaptionFormat{custom}
{%
  #1#2 #3
}
\captionsetup
{
  format=custom,%
  labelformat=custom,%
  labelsep=custom
}

\begin{document}

\begin{center}
  {\Large\bfseries Math 342 Tutorial} \\
  {\normalsize\bf June 18, 2025}
\end{center}

Recall an algebraic monoid is a set $G$ together with a binary operation $G
\times G \rightarrow G$ satisfying the following:
\begin{itemize}
\item[{\bf M1}] Associativity: $a(bc) = (ab)c$ for all $a,\,b,\,c \in G$.
\item[{\bf M2}] Identity: There is an element $e \in G$ for which $ea=ae=a$ for
  all $a \in G$.
\end{itemize}

An example of a monoid is the set of all $n \times n$ matrices with entries over
$\C$.

An algebraic group is a monoid $G$ satisfying the additional axiom:
\begin{itemize}
\item[{\bf G1}] Inverse: For every $a \in G$, there is a $b \in G$ for which
  $ab=ba=e$.
\end{itemize}

An example of a group is the set of all $n \times n$ matrices over $\C$ which
are invertible.

The reader is also reminded of the following definition used previously. If
$f,\,g$ are two arithmetic functions, then their Dirchlet product $f * g$ is
defined as
\[
  (f * g)(n) = \sum_{d \mid n}f(n)g\left( \frac{n}{d} \right).
\]

\noindent{\bf Question 1.} For this question, you will need to use results we
have proven in previous tutorials. Do the following.
\begin{enumerate*}[{\bf (a)}]
\item Prove the set $\mathscr{F}$ of all arithmetic functions is a monoid. What
  is the identity element of $\mathscr{F}$?
\item What is the largest subset $\mathscr{G}$ of $\mathscr{F}$ such that
  $\mathscr{G}$ is a group?
\item Name a second subset $\mathscr{M} \neq \mathscr{G}$ of $\mathscr{F}$ for
  which $\mathscr{M}$ is also a group.
\end{enumerate*} \\

\noindent{\bf Question 2.} Do the following.
\begin{enumerate*}[{\bf (a)}]
\item Prove the M\"{o}bius inversion formula: Given $f(n) = \sum_{d \mid
    n}g(d)$, one has $g(n) = \sum_{d \mid n}f(d)\mu(\frac{n}{d})$.
\item We previously established that $\phi(n) = \sum_{d \mid
    n}d\mu(\frac{n}{d})$. Use part {\bf (a)} to show that $n=\sum_{d \mid
    n}\phi(d)$.
\end{enumerate*} \\

\noindent{\bf Question 3.} The Mangolt function $\Lambda(n)$ is defined as
\[
  \Lambda(n) =
  \begin{cases}
    \log p & \text{if $n=p^a$ for some prime $p$,} \\
    0 & \text{otherwise.}
  \end{cases}
\]

\noindent Show the following.
\begin{enumerate*}[{\bf (a)}]
\item If $n \geqq 1$, then $\log n = \sum_{d \mid n}\Lambda(d)$.
\item Use part {\bf (a)} to show that $\Lambda(n) =$ $ \sum_{d \mid
    n}\mu(d)\log\frac{n}{d} = - \sum_{d \mid n}\mu(d)\log d$.
\end{enumerate*} \\

An arithmetic function $f$ is completely multiplicative if $f(mn)=f(m)f(n)$ for
all $m,\,n \in \Z$ (we drop the condition that $(m,\,n)=1$). \\

\noindent{\bf Question 4.} Let $f$ be an arithmetic function. Show the
following.
\begin{enumerate*}[{\bf (a)}]
\item Let $f$ be multiplicative. Then $f$ is completely multiplicative if and
  only if $f^{-1}=\mu f$, where $f^{-1}$ is the Dirichlet inverse of $f$.
\item If $f$ is multiplicative, then $\sum_{d \mid n}\mu(d)f(d) = \prod_{p \mid
    n}(1-f(p))$.
\end{enumerate*} \\

\noindent{\bf Question 5.} If $n=p_1^{e_1} \cdots p_k^{e_k}$, then Liouville's
function $\lambda$ is defined as
\[
  \lambda(n) = (-1)^{e_1 + \cdots + e_k}.
\]
Show the following.
\begin{enumerate*}[{\bf (a)}]
\item $\lambda$ is completely multiplicative.
\item $\lambda(n) = 1$ if $n$ is a square and 0 otherwise. Also,
  $\lambda^{-1}=\abs{\mu}$.
\end{enumerate*} \\

\noindent{\bf Question 6.} Show the following identities.
\begin{enumerate}[{\bf (a)}]
\item
  \[
    \frac{n}{\phi(n)} = \sum_{d \mid n}\frac{\mu^2(d)}{\phi(d)}.
  \]
  [Hint: Show first that $\frac{\mu^2(n)}{\phi(n)}$ is multiplicative.]
\item Show for each $k \geqq 1$ that
  \[
    \sum_{\begin{smallmatrix}d \mid n \\ d^k \mid n\end{smallmatrix}}\mu(d) =
    \begin{cases}
      0 & \text{if $m^k \mid n$ for some $m>1$,} \\
      1 & \text{otherwise.}
    \end{cases}
  \]
\end{enumerate}

\end{document}

